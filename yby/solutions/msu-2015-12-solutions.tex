\documentclass[11pt, a4paper]{article}

\usepackage[T2A]{fontenc}		%cyrillic output
\usepackage[utf8]{inputenc}		%cyrillic output
\usepackage[english, russian]{babel}	%word wrap
\usepackage{amssymb, amsfonts, amsmath}	%math symbols
\usepackage{mathtext}			%text in formulas
\usepackage{geometry}			%paper format attributes
\usepackage{fancyhdr}			%header
\usepackage{graphicx}			%input pictures
\usepackage{tikz}				%draw pictures
\usetikzlibrary{patterns}		%draw pictures: fill
\usetikzlibrary{calc}			%draw pictures: coordinate calc
% \usetikzlibrary{external}		%draw pictures: cache pitcures
% \tikzexternalize				%cache pictures
\usepackage{listings}			%code
\usepackage{enumitem}			%noindent

\geometry{left=1cm, right=1cm, top=2cm, bottom=1cm, headheight=28pt}
\setlist[enumerate]{leftmargin=*}	%remove enumarate indenttion
\sloppy							%correct overfull
\newcommand{\subsectionbreak}{\clearpage} %new section

\newcommand{\head}[4]
{
	\pagestyle{fancy}
	\fancyhf{}
	\lhead{#3 \\ #2}
	\rhead{#1}
	
	\begin{center}
	\begin{large}
	#1 \\
	\textit{#2}\\
	\end{large}
	\end{center}
}

\lstset{language=C++}
\lstset{basicstyle=\footnotesize\ttfamily} 
\lstset{keywordstyle=\color{blue}}
\lstset{frame=single}
\lstset{numbers=left}

\sloppy

\newcommand{\problemauthor}[1]{
\begin{flushright}
\textit{Автор: #1}
\end{flushright}
}

\newcommand{\problemofferer}[1]{
\begin{flushright}
\textit{Предложил: #1}
\end{flushright}
}


\begin{document}

\head{Открытая олимпиада по программированию \\ Зимний тур 2015}{18 декабря 2015}{Казахстанский филиал МГУ имени М.В.Ломоносова}{г.~Астана}

\subsubsection*{A. Alexandra and Exam} 

\problemauthor{Жусупов Али}

Вычислим четыре значения $x \pm y \pm z$ и прибавим к ответу те из них, которые делятся на 10.

Асимптотика: $O(1)$.



\subsubsection*{B. Ball and Snowy Cube} 

\problemauthor{Журавская Александра}

Если сфера целиком лежит в одном из октантов ($|x| \geqslant r$, $|y| \geqslant r$ и $|z| \geqslant r$), то минимальная возможная сторона куба равна
$$\max(|x| + r, |y| + r, |z| + r).$$
В противном случае, такого куба не существует.

Асимптотика: $O(1)$.



\subsubsection*{С. Calculation of Erulan Numbers} 

\problemauthor{Абайулы Ерулан}

Легко найти, что при $k$ равном 1, 2, 3 и 4 подходящих чисел будет $num[k]:$ 9, 14, 56 и 260, соответственно. Что сразу позволяет отсеять неподходящие варианты, если $n > num[k]$.

При $k < 6$ ответы можно сгенерировать вручную, проверяя каждое число. При $k = 6$ подходящих чисел набирается достаточно среди чисел одного из двух видов: из цифр 1, 2, 4 и 8 и из цифр 1 и 5. В частности, 1000-ное число такого вида равно 884488 (если оставить только числа первого вида, то их будет всего 984, что недостаточно для крайнего случая). Таким образом, можно сгенерировать 1000 подходящих шестизначных чисел.

Заметим, что при добавлении ведущих единиц, делимость на 1, 2, 4, 5 и 8 не изменяется. Соответственно, при $k > 6$, можно дописать к полученным числам слева $(k-6)$ единиц.

Асимптотика: $O(n k)$.



\subsubsection*{D. Do Rain Dance} 

\problemauthor{Жусупов Али}

$$\sum_{i=1}^{N} \sum_{j=1}^{M} (i + j^2) = \frac{m n (n+1)}{2} + \frac{n m (m+1) (2m+1)}{6}.$$

Асимптотика: $O(1)$.

Замечание: наивное решение не проходит ограничения по времени.



\subsubsection*{E. ECM and Cognitive Dissonance} 

\problemofferer{Камалбеков Тимур}

Если $k > \frac{n (n-1)}{2}$, то такой перестановки не существует. А если $k < n$, то достаточно расставить все числа по возрастанию от 2 до $n$, а число 1 поставить между $k$-м и $(k+1)$-м числом. 

Иначе будем размещать числа от 1 до $n$ в массиве с конца в начало ($a[n + 1 - i] = i$) до тех пор, пока количество инверсий, которые образуют уже расставленные числа со всеми остальными и между собой не превосходит $k$. Это легко сделать, так как число $i$ на позиции $n + 1 - i$ образует $n-i$ инверсию c числами, которые стоят на позициях с 1 по $n - i$. После того, как данные числа будут расставлены останутся $t$ чисел, на которых останется менее $t$ инверсий между этими $t$ числами. Достаточно расставить их также как и в случае $k < n$.

Асимптотика: $O(n)$.



\subsubsection*{F. Fibonacci and Prime Number} 

\problemauthor{Жусупов Али, Васильев А.Н.}

Достаточно заметить, что простых чисел в диапазоне от 1 до $N$ порядка $O(N / \log{N})$ (при указанных ограничениях задачи их около $4\cdot 10^{10}$). В то время как чисел Фибоначчи порядка $O(\log{N})$ (при указанных ограничениях задачи их не более 60). Поэтому разумно перебирать числа Фибоначчи $F_k$, которые не превосходят $n$ и проверять, является ли число $N - F_k$ простым наивным способом (делители до корня из числа).

Асимптотика: $O(\sqrt{N} \log{N})$.

Замечание: перебор всех простых от 1 до $N$ не укладывается по ограничениям времени.



\subsubsection*{G. Game of Castles} 

\problemauthor{Камалбеков Тимур}

В задаче необходимо было промоделировать  то, что написано в условии. Для этого достаточно хранить статус ($status[k] = 1$, если $k$-й замок уставший и $status[k] = 0$ --- иначе). Если при ходе $i \rightarrow j$ известно, что $a_i > 0$, то необходимо уменьшить здоровье $j$-го на $1 - status[i] + status[j]$.

Асимптотика: $O(N + M)$.

\end{document} 
