\documentclass[11pt, a4paper]{article}

\usepackage[T2A]{fontenc}		%cyrillic output
\usepackage[utf8]{inputenc}		%cyrillic output
\usepackage[english, russian]{babel}	%word wrap
\usepackage{amssymb, amsfonts, amsmath}	%math symbols
\usepackage{mathtext}			%text in formulas
\usepackage{geometry}			%paper format attributes
\usepackage{fancyhdr}			%header
\usepackage{graphicx}			%input pictures
\usepackage{tikz}				%draw pictures
\usetikzlibrary{patterns}		%draw pictures: fill
\usetikzlibrary{calc}			%draw pictures: coordinate calc
% \usetikzlibrary{external}		%draw pictures: cache pitcures
% \tikzexternalize				%cache pictures
\usepackage{listings}			%code
\usepackage{enumitem}			%noindent

\geometry{left=1cm, right=1cm, top=2cm, bottom=1cm, headheight=28pt}
\setlist[enumerate]{leftmargin=*}	%remove enumarate indenttion
\sloppy							%correct overfull
\newcommand{\subsectionbreak}{\clearpage} %new section

\newcommand{\head}[4]
{
	\pagestyle{fancy}
	\fancyhf{}
	\lhead{#3 \\ #2}
	\rhead{#1}
	
	\begin{center}
	\begin{large}
	#1 \\
	\textit{#2}\\
	\end{large}
	\end{center}
}

\lstset{language=C++}
\lstset{basicstyle=\footnotesize\ttfamily} 
\lstset{keywordstyle=\color{blue}}
\lstset{frame=single}
\lstset{numbers=left}

\sloppy

\newcommand{\problemauthor}[1]{
\begin{flushright}
\textit{Автор: #1}
\end{flushright}
}

\newcommand{\problemofferer}[1]{
\begin{flushright}
\textit{Предложил: #1}
\end{flushright}
}


\begin{document}

\head{Открытая олимпиада по программированию \\ Весенний тур 2016}{30 апреля 2016}{Казахстанский филиал МГУ имени М.В.Ломоносова}{г.~Астана}

\subsubsection*{A. Alexandra’s subtractions} 

\problemauthor{ Абдикалыков А.К.}

Вторая максимальная разность равна: $d_2 = \max(a_n - a_2, a_{n-1} - a_1)$.

Асимптотика: $O(n)$.

Замечание: прямой перебор $O(n^2)$ не проходит по ограничениям времени.



\subsubsection*{B. Book of all the words} 

\problemauthor{ Абдикалыков А.К.}

Перевести число $(k-1)$ в $n$-ричную систему счисления.

Асимптотика: $O(m)$.

Замечание: прямой перебор $O(n^m)$ не проходит по ограничениям времени.



\subsubsection*{С. Change the word} 

\problemauthor{ Баев А.Ж.}

Если запретить операцию изменения букв строки на симметричные, то ответ будет:
$$f(s_1, s_2) = len(s_1) + len(s_2) - 2 * len(lcs(s_1, s_2)),$$
где $len(a)$ --- длина строки $a$, $lcs(a, b)$ --- наибольшая общая подстрока строк $a$ и $b$ (считается динамическим программированием за $O(len(a) * len(b))$. 

Если вернуть операцию изменения строк, то ясно, что она применяется не более одного раза. Тогда ответ на задачу:
$$ \max \left( f(s_1, s_2), f(s_1, \overline{s}_2) + 1 \right),$$
где $\overline{s}$ --- строка, полученная из строки $s$ заменой всех букв на симметричные.

Асимптотика: $O(len(s_1)  len(s_2) )$.



\subsubsection*{D. Doubtful numbers} 

\problemauthor{ Абдикалыков А.К.}

Легко понять, что подходят только числа вида $pq$, где $p$ и $q$ --- различные простые числа. Обозначим через $z(n)$ --- количество чисел такого вида среди чисел от 1 до $n$ (тогда ответ вычисляется как $z(b) - z(a-1)$). Фиксируем простое $p$. Подходящих чисел вида $pq$, где $p < q$ и $pq \le n$ будет: $ \pi\left( \left[ \frac{n}{p} \right] \right) - \pi(p)$. Заметим, что перебирать данные $p$ достаточно до $\sqrt{n}$:

$$z(n) = \sum_{p \le \sqrt{n}}^{n} \pi\left(\left[ \frac{n}{p} \right] \right) - \pi(p)$$.

Решение состоит из 3 частей: построение всех простых от 1 до $n$ ($O(n \log \log{n})$ --- решето Эратосфена ), просчет $\pi(k)$ за $O(n)$ действий и вычисление $z(n)$ за $O(\sqrt{n})$. 

Вместо вычисления функции $pi(n)$ можно вычислить данную сумму с помощью 2 указателей ($p$ --- первый указатель, двигающийся от 1 до $n$, $q$ --- второй указатель, двигающийся от $n$ до 1).

Асимптотика: $O(n \log \log{n})$. 

Замечание 1: решения с асимптотикой $O(n \sqrt{n})$ и хуже не проходят по ограничениям времени.

Замечание 2: решения с неэффективным построением решета (например 2 целочисленных массива размера $n$) не проходят по ограничениям памяти.

Замечание 3: данную задачу можно решить с асимптотикой $O( \sqrt{n} )$, используя подход Генри Лемера для метода Мейсселя.



\subsubsection*{E. Experiment with tea} 

\problemauthor{ Баев А.Ж.}

Зафиксируем некоторый уровень воды $h$. Построим компоненту связности, начиная с нулевой по высоте клетки, на таблице со следующим условия связности: из клетки $(i_1, j_1)$ можно попасть в $(i_2, j_2)$, если эти клетки соседние по стороне и $a_{i_2 j_2} < h$. Обход можно произвести с помощью алгоритма поиска в ширину или в глубину. При этом, если мы выходим на граничные клетки, то считаем, что обход завершился переполнением. Задача сводится к поиску максимального $h$, при котором обход не завершается переполнением, что легко решается бинарным поиском по $h$.

Асимптотика: $O(n^2 \log(h) )$. 

Замечание:  решения с асимптотикой $O(n^2 h)$ и хуже не проходят по ограничениям времени.



\subsubsection*{F. Five words} 

\problemauthor{ Абдикалыков А.К., Баев А.Ж.}

Подсказка 1: <<Коровы понимают каждое сказанное человеком слово, правда, только частично>>.

Подсказка 2: первые буквы каждой строки (КФМГУ).

Ответом является подстрока длины 3 строки:\\ moscowstateuniversitykazakhstanbranch



\subsubsection*{G. Glowing letters} 

\problemauthor{ Абдикалыков А.К.}

Для строки длины $m$ без букв 'a' количество подстрок равно
$$\frac{m(m+1)}{2},$$
а количество подпоследовательностей
$$2^m - 1.$$

Пусть буквы 'a' делят исходную строку на подстроки без букв 'a' с длинами $m_1$, $m_2$, ..., $m_k$. Тогда количество подстрок равно
$$\sum_{i=1}^k \frac{m_i(m_i+1)}{2},$$
а количество подпоследовательностей
$$2^{\sum_{i=1}^{k} m_i} - 1.$$ 

Асимптотика: $O(n)$.

Замечание 1: решения с асимптотикой $O(n^2)$ и хуже не проходят по ограничениям времени.

Замечание 2: решения с неаккуратным взятием ответа по модулю не проходят некоторые тесты. Например следующий код дает неверный ответ:
$$ans := (ans + m[i] * m[i+1]) \mod 1000000007.$$



\subsubsection*{H. Harmonic permutations} 

\problemauthor{ Абдикалыков А.К.}

Решим эту задачу в общем виде, а именно найдём число $d(s, t)$ перестановок длины $s + t$ таких, что:
\begin{enumerate}
\item $a_1 < a_2 < \dots < a_s$ ;
\item $a_{s+1} < a_{s+2} < \dots < a_{s+t}$ ;
\item $a_i < a_{s+i}$, $\forall i = 1, 2, . . . , min(s, t)$.
\end{enumerate}
Тогда ответом будет число $d(n, n)$.

Чтобы найти $d(s, t)$, определим положение числа $s + t$. Если $s \le t$, то возможен только один вариант: $a_{s+t} = s + t$. Если $s > t$, то появляется второй вариант $a_s = s + t$. Таким образом определяется рекуррентная формула:
$$
d(s, t) =
\begin{cases}
d(s, t-1) + d(s-1, t), & s > t,\\
d(s, t-1), & s \le t.\\
\end{cases}
$$

Асимптотика: $O(n^2)$. 

Замечание 1: решения с асимптотикой $O(n^3)$ и хуже не проходят по ограничениям времени.

Замечание 2: если заметить соответствие строящейся перестановки с правильными скобочными структурами, то можно понять, что ответом будут числа Каталана, соответственно существует решение за $O(n)$.



\subsubsection*{I. Infinity problem} 

\problemauthor{ Абдикалыков А.К.}

Числа с суммой цифр равной 3 представляются в виде: $10^b + 10^c + 10^d$. Сгенерируем все остатки $10^i \; \mod n$; их будет не более $n$ штук (для данных ограничений, можно убедиться, что их не более 2000): $rem[t] = 1$, если остаток $t$ был сгенерирован, и $rem[t] = 0$ иначе. Далее достаточно перебрать все возможные остатки $t_1$ и $t_2$, которые сгенерированы и проверить наличие остатка $n - t_1 - t_2$.

Асимптотика: $O(n^2)$.

Замечание 1: решения с асимптотикой $O(n^3)$ и хуже не проходят по ограничениям времени.

Замечание 2: заметим, что при $n$ взаимно простым с 10 число $10^b + 10^c + 10^d$ делится на $n$ только если $10^{b-d} + 10^{c-d} + 1$. Соответственно существует решение за $O(n)$.



\subsubsection*{J. Jelly cake} 

\problemauthor{ Баев А.Ж.}

Легко доказывается, что вершины треугольника с минимальной площадью будут соседними. Поэтому сортируем все точки по их углам $\varphi_i$, перебираем все тройки соседних точек и выбираем минимальную. Площадь каждого такого треугольника можно найти по формуле:
$$2 R^2 sin(\alpha) sin(\beta) sin(\gamma),$$
где $\alpha = \frac12 (\varphi_{i+1} - \varphi_{i})$, $\beta = \frac12 (\varphi_{i+2} - \varphi_{i+1})$, $\gamma= \frac12 (\varphi_{i+2} - \varphi_{i})$ и углы $\varphi_i$ зациклены (то есть $\varphi_{n+k} = \varphi_{k}$).

Асимптотика: $O(n \log n)$.

Замечание 1: решения с асимптотикой $O(n^2)$ и хуже не проходят по ограничениям времени.

Замечание 2: решение на типе $float$ на языке $C$ получает неверный ответ из-за ошибок округления.

\end{document} 
