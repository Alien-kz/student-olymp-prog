\documentclass[12pt, a4paper]{article}

\usepackage[T2A]{fontenc}		%cyrillic output
\usepackage[utf8]{inputenc}		%cyrillic output
\usepackage[english, russian]{babel}	%word wrap
\usepackage{amssymb, amsfonts, amsmath}	%math symbols
\usepackage{mathtext}			%text in formulas
\usepackage{geometry}			%paper format attributes
\usepackage{fancyhdr}			%header
\usepackage{graphicx}			%input pictures
\usepackage{tikz}				%draw pictures
\usetikzlibrary{patterns}		%draw pictures: fill
\usetikzlibrary{calc}			%draw pictures: coordinate calc
% \usetikzlibrary{external}		%draw pictures: cache pitcures
% \tikzexternalize				%cache pictures
\usepackage{listofitems}		%list of arguments (pictures)
\usepackage{enumitem}			%enumarate parameters
\usepackage{titlesec}			%new subsection option


\geometry{left=1cm, right=1cm, top=2cm, bottom=1cm, headheight=15pt}
\setlist[enumerate]{leftmargin=*}	%remove enumarate indenttion
\sloppy							%correct overfull
\newcommand{\subsectionbreak}{\clearpage} %new section

\newcommand{\head}[4]
{
	\pagestyle{fancy}
	\fancyhf{}
	\lhead{#3 \\ #2}
	\rhead{#1}
}


% format

\newcommand{\informat}[1]
{
	\paragraph{Ввод.\\} #1
}

\newcommand{\outformat}[1]
{
	\paragraph{Вывод.\\} #1
}

\newcommand{\example}[2]
{
	\paragraph{Пример.\\}
	{\tt
	\begin{tabular}{|p{0.4\linewidth}|p{0.4\linewidth}|}
	\hline
	Ввод & Вывод \\
	\hline
	#1 & #2		\\
	\hline
	\end{tabular}
	}
}

\newcommand{\examplee}[4]
{
	\paragraph{Пример.\\}
	{\tt
	\begin{tabular}{|p{0.4\linewidth}|p{0.4\linewidth}|}
	\hline
	Ввод 	& Вывод  	\\
	\hline
	#1 		& #2 		\\
	\hline
	#3		& #4		\\
	\hline
	\end{tabular}
	}
}

\newcommand{\examplEEE}[6]
{
	\paragraph{Пример.\\}
	{\tt
	\begin{tabular}{|p{0.5\linewidth}|p{0.3\linewidth}|}
	\hline
	Ввод 	& Вывод  	\\
	\hline
	#1 		& #2 		\\
	\hline
	#3		& #4		\\
	\hline
	#5		& #6		\\
	\hline
	\end{tabular}
	}
}

\newcommand{\exampleee}[6]
{
	\paragraph{Пример.\\}
	{\tt
	\begin{tabular}{|p{0.4\linewidth}|p{0.4\linewidth}|}
	\hline
	Ввод 	& Вывод  	\\
	\hline
	#1 		& #2 		\\
	\hline
	#3		& #4		\\
	\hline
	#5		& #6		\\
	\hline
	\end{tabular}
	}
}

\newcommand{\exampleeee}[8]
{
	\paragraph{Пример.\\}
	{\tt
	\begin{tabular}{|p{0.4\linewidth}|p{0.4\linewidth}|}
	\hline
	Ввод 	& Вывод  	\\
	\hline
	#1 		& #2 		\\
	\hline
	#3		& #4		\\
	\hline
	#5		& #6		\\
	\hline
	#7		& #8		\\
	\hline
	\end{tabular}
	}
}

\newcommand{\exampleeeee}[5]
{
	\paragraph{Пример.\\}
	{\tt
	\begin{tabular}{|p{0.4\linewidth}|p{0.4\linewidth}|}
	\hline
	Ввод 	& Вывод  	\\
	\hline
	#1		\\
	\hline
	#2		\\
	\hline
	#3		\\
	\hline
	#4		\\
	\hline
	#5		\\
	\hline
	\end{tabular}
	}
}

\newcommand{\examplepic}[3]
{
	\subsection*{Пример.}
	{\tt
	\noindent
	\begin{tabular}{|p{0.1\linewidth}|p{0.1\linewidth}|p{0.5\linewidth}|}
	\hline
	Ввод 	& Вывод  	& Пояснение\\
	\hline
	#1 		& #2 		& #3\\
	\hline
	\end{tabular}
	}
}


\newcommand{\excomm}[1]
{
	\paragraph{Комментарий. \\}
	\textit{#1}
}

\begin{document}

\head{Открытая олимпиада по программированию \\ Зимний тур 2017}{19 декабря 2017}{Казахстанский филиал МГУ имени М.В.Ломоносова}{г.~Астана}

\subsection*{A. Askhana}


Рамазан любит кушать в столовой. После долгого размышления на тему о правильном питании, он пришел к выводу, что каждый день будет брать одно из 3 блюд. При этом блюда он выбирает поочередно: первое, второе, третье, первое, второе, третье и т.д. Чувство удовлетворённости от каждого обеда равно $(1000 Q - P)$, где $P$ --- цена блюда (установлена столовой), $Q$ --- жирность блюда (установлена Рамазаном опытным путём). Рамазан посетил столовую уже $N$ раз. Чему равно суммарное чувство удовлетворенности от всех обедов?

\informat{В первой строке целое число $N$ от 1 до $10^9$ --- число обедов. \newline Далее \textbf{три} строки по два числа: цена блюда $P$ (от 100 до 1000) и жирность блюда $Q$ от 1 до 5.}

\outformat{Одно целое число --- чувство удовлетворенности за $N$ дней.}

\examplee
{5 \newline
100 5 \newline
150 4 \newline
250 3}
{20250}
{3 \newline
200 1 \newline
400 2 \newline
700 5}
{6700}



\subsection*{B. Binecraft}


Грамотно распределяя деньги со стипендии, Жалгас купил машину, правда пока только в компьютерной игре Binecraft. Особенность игры в том, что вся поверхность мира, и даже парковка, поделена на квадратные клетки размера~1. Теперь Жалгасу предстоит выбрать парковочное место под размер своей машины (ни больше, ни меньше). Строгие правила парковки разрешают ставить машину только по вертикали или горизонтали, занимая несколько целых клеток. При этом часть клеток парковки уже занята другими объектами игровой Вселенной. Сколькими способами Жалгас может выбрать себе парковочное место?

\informat{В первой строке даны два целых числа $N$ и $M$ от 1 до 1000 --- размеры машины. \newline 
Во второй строке даны два целых числа $H$ и $W$ от 1 до 1000 --- размеры парковки.\newline 
Далее в $H$ строках описание парковки размера $H \times W$, где {\tt .} (точка)  соответствует свободной клетке, а {\tt \#} (решётка) --- занятой.}

\outformat{Одно целое число --- количество различных способов выбрать парковочное место.}

\examplee
{2 3 \newline
4 5 \newline
..\#.. \newline
..\#.. \newline
..... \newline
.....}
{7}
{3 3 \newline
3 6 \newline
...... \newline
...\#\#\# \newline
......}
{1}



\subsection*{C. Course}


У группы мехмата начался блок, который проходит в аудитории $603$. Места в этой аудитории можно описать в виде матрицы $6 \times 6$. Парты, расположенные на одинаковом расстоянии от доски, образуют ряд. В каждом ряду есть 6 мест, которые определяются номером варианта от 1 до 6 (слева направо).

Батима, как староста, следит за временем прихода одногруппников. Например, в первый день блока $i$-й студент группы пришел в момент времени $A_i$. Батима заметила, что с каждым днем её одногруппники приходят всё позже и позже, а именно время прихода $i$-го студента в $j$-й день увеличивается на $P_{ij}$ минут по сравнению с предыдущим днём. Студенты всегда хотят сесть как можно ближе к доске. Когда студент заходит в аудиторию, он ищет наиболее близкое к доске место. А если таких мест в ряду несколько, то студент выбирает место с минимальным вариантом. В случае, когда время прихода у двух или нескольких студентов одинаково, то сначала место занимает студент с меньшим номером. 

Батима записала некоторые интересные данные: кто сидел в день $d$ на $r$-м ряду на варианте $c$. Можете ли Вы восстановить эту информацию?

\informat{В первой строке дано два целых числа: $N$ от 1 до 36 --- количество студентов и $D$ от 1 до 100000 --- длительность блока в днях. \newline 
Далее $N$ строк, в $i$-й строке ($i$ --- номер студента по жураналу от 1 до $N$) записано число $A_i$ от 0 до $10^7$ и $(D - 1)$ число $P_{ij}$ от 0 до $10^7$ ($j$ --- номер дня от 2 до $D$). \newline 
Далее число $M$ от 1 до 100000 --- число запросов.  \newline
Далее $M$ троек чисел: $d_k$ от 1 до $D$ --- номер дня, $r_k$ и $c_k$ от 1 до 6 --- номер ряда и варианта соответственно ($k$ от 1 до $M$).}

\outformat{$M$ целых чисел --- номер студента, сидящего в $d_k$-й день на ряду $r_k$ за вариантом $c_k$ ($k$ от 1 до $M$).}

\example
{
4 3 \newline
1 3 2 \newline
3 1 1 \newline
3 0 3 \newline
3 1 2 \newline
2 \newline
2 1 1 \newline
3 1 4
}
{
3 \newline
4
}

\excomm{В первый день студенты сидели так: \newline
\begin{tabular}{cccccc}
1 2 3 4 0 0 \\
0 0 0 0 0 0 \\
0 0 0 0 0 0 \\
0 0 0 0 0 0 \\
0 0 0 0 0 0 \\
0 0 0 0 0 0 \\
\end{tabular}
\newline
В второй день студенты сидели так: \newline
\begin{tabular}{cccccc}
3 1 2 4 0 0 \\
0 0 0 0 0 0 \\
0 0 0 0 0 0 \\
0 0 0 0 0 0 \\
0 0 0 0 0 0 \\
0 0 0 0 0 0 \\
\end{tabular}
\newline
В третий день студенты сидели так: \newline
\begin{tabular}{cccccc}
2 1 3 4 0 0 \\
0 0 0 0 0 0 \\
0 0 0 0 0 0 \\
0 0 0 0 0 0 \\
0 0 0 0 0 0 \\
0 0 0 0 0 0 \\
\end{tabular}
}


\subsection*{D. Decoration}


Димитрий, Павел и Куат готовятся к новому году. Они купили $N$ коробок, в $i$-й из них находится $A_i$ новогодних шаров, которыми они хотят нарядить $K$ ёлок. При этом на всех ёлках должно быть одинаковое количество шаров. Сколько у них способов выбрать несколько подряд стоящих коробок, чтобы суммарное количество шаров в этих коробках делилось на $K$?

\informat{В первой строке целое число $N$ от 1 до $10^5$ и $K$ от 1 до $10^5$. \newline
Во второй строке $N$ целых чисел от 0 до $10^{18}$.}

\outformat{Одно целое число --- количество различных способов выбрать подряд стоящие коробки.}

\examplee
{5 3 \newline
1 2 3 4 5}
{7}
{6 2 \newline
2 2 2 1 1 1}
{11}

\excomm{В первом примере подходят варианты: \newline 1+2, 3, 1+2+3, 4+5, 3+4+5, 1+2+3+4+5, 2+3+4.}



\subsection*{E. Easy shifting}


Азат недавно познакомился с инверсиями. Он взял некоторую перестановку чисел от 1 до $N$ и посчитал количество инверсий. Но данное количество ему показалось недостаточно большим. Какое максимальное число инверсий Азат может получить, если он умеет циклически сдвигать исходную перестановку?

\informat{В первой строке одно целое число $N$ от 1 до $10^5$. \newline 
Во второй строке $N$ целых чисел --- перестановка чисел от 1 до $N$.}

\outformat{Одно целое число --- максимальное число инверсий.}

\examplee
{5 \newline
1 4 2 3 5}
{6}
{6 \newline
1 3 4 5 6 2}
{10}

\excomm{В первом примере максимальное количество инверсий будет в перестановке: 3 5 1 4 2.}



\subsection*{F. Fix position}


Ануар ответственный за рассадку участников олимпиады. В компьютерном кабинете компьютеры стоят в два параллельных ряда, между которым ровно 1 метр. Ануар заметил, что участники, сидящие в одном ряду, не могут списать друг у друга. А участники, сидящие в разные рядах, могут, причем, чем меньше расстояние между ними, тем больше шансов списать. В связи с этим, Ануар хочет узнать минимальное расстояние между компьютерами из разных рядов. Помогите ему!

\informat{В первой строке вводится одно целое число $N$ от 1 до $10^5$. \newline 
Во второй строке $N$ целых чисел от $-10^9$ до $10^9$ --- координаты компьютеров по оси $OX$ первого ряда (координата по оси $OY$ у всех равна 0). \newline
В третьей строке вводится одно целое число $M$ от 1 до $10^5$. \newline
Во четвертой строке $M$ целых чисел от $-10^9$ до $10^9$ --- координаты компьютеров по оси $OX$ второго ряда (координата по оси $OY$ у всех равна 1).}

\outformat{Одно целое число --- минимальное расстояние с точностью 4 знака после запятой.}

\examplee
{3 \newline
1 2 3 \newline
3 \newline
3 4 5}
{1.0000}
{6 \newline
1 2 3 6 5 4 \newline
1 \newline
0}
{1.4142}



\subsection*{G. Graphland}


В вымышленной стране Бекарыса, Графландии, есть $N$ городов с номерами от 1 до $N$, которые образуют области. Областью называется такое множество городов, что из любого города области по дорогам можно добраться до любого другого города области. Бекарыс решил объединить две какие-то области, построив одну дополнительную дорогу. Стоимость строительства дороги между городами $i$ и $j$ равна $a_{ij}$ условных единиц, причем далеко не все пары городов можно соединить. Какую наиболее дешевую дорогу он сможет построить?

\informat{В первой строке два целых числа $N$ от 1 до $10^5$ и $M$ от 1 до $2 \cdot 10^5$. \newline
Далее каждая из $M$ строк содержит два числа $i$ и $j$ от 1 до $N$, задающие дорогу между городами $i$ и $j$. \newline 
Далее одно целое число $T$ от 1 до $2 \cdot 10^5$. \newline
Далее каждая из $T$ строк содержит три числа $i$, $j$ от 1 до $N$ и $a_{ij}$ от 1 до $10^9$, определяющие возможную дорогу между городами $i$ и $j$, которую можно построить за $a_{ij}$ условных единиц. Гарантируется, что каждая пара городов встречается во входных данных не более одного раза.}

\outformat{Два числа $i$, $j$ ($i < j$) --- номера городов, между которым следует построить дорогу. Если ответов несколько, выведите тот, для которого $i$ минимален. Если же таких несколько, выведите тот для которого y минимален. В случае, если такой дороги не существует, выведите $-1$ (минус один).}

\examplee
{3 1 \newline
1 2 \newline
2 \newline
1 3 5 \newline
3 2 4
}
{2 3}
{4 4 \newline
1 3 \newline
1 4 \newline
2 3 \newline
2 4 \newline
2 \newline
1 2 5 \newline
3 4 2}
{-1}


\end{document} 
