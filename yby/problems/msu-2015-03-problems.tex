\documentclass[12pt, a4paper]{article}

\usepackage[T2A]{fontenc}		%cyrillic output
\usepackage[utf8]{inputenc}		%cyrillic output
\usepackage[english, russian]{babel}	%word wrap
\usepackage{amssymb, amsfonts, amsmath}	%math symbols
\usepackage{mathtext}			%text in formulas
\usepackage{geometry}			%paper format attributes
\usepackage{fancyhdr}			%header
\usepackage{graphicx}			%input pictures
\usepackage{tikz}				%draw pictures
\usetikzlibrary{patterns}		%draw pictures: fill
\usetikzlibrary{calc}			%draw pictures: coordinate calc
% \usetikzlibrary{external}		%draw pictures: cache pitcures
% \tikzexternalize				%cache pictures
\usepackage{listofitems}		%list of arguments (pictures)
\usepackage{enumitem}			%enumarate parameters
\usepackage{titlesec}			%new subsection option


\geometry{left=1cm, right=1cm, top=2cm, bottom=1cm, headheight=15pt}
\setlist[enumerate]{leftmargin=*}	%remove enumarate indenttion
\sloppy							%correct overfull
\newcommand{\subsectionbreak}{\clearpage} %new section

\newcommand{\head}[4]
{
	\pagestyle{fancy}
	\fancyhf{}
	\lhead{#3 \\ #2}
	\rhead{#1}
}


% format

\newcommand{\informat}[1]
{
	\paragraph{Ввод.\\} #1
}

\newcommand{\outformat}[1]
{
	\paragraph{Вывод.\\} #1
}

\newcommand{\example}[2]
{
	\paragraph{Пример.\\}
	{\tt
	\begin{tabular}{|p{0.4\linewidth}|p{0.4\linewidth}|}
	\hline
	Ввод & Вывод \\
	\hline
	#1 & #2		\\
	\hline
	\end{tabular}
	}
}

\newcommand{\examplee}[4]
{
	\paragraph{Пример.\\}
	{\tt
	\begin{tabular}{|p{0.4\linewidth}|p{0.4\linewidth}|}
	\hline
	Ввод 	& Вывод  	\\
	\hline
	#1 		& #2 		\\
	\hline
	#3		& #4		\\
	\hline
	\end{tabular}
	}
}

\newcommand{\examplEEE}[6]
{
	\paragraph{Пример.\\}
	{\tt
	\begin{tabular}{|p{0.5\linewidth}|p{0.3\linewidth}|}
	\hline
	Ввод 	& Вывод  	\\
	\hline
	#1 		& #2 		\\
	\hline
	#3		& #4		\\
	\hline
	#5		& #6		\\
	\hline
	\end{tabular}
	}
}

\newcommand{\exampleee}[6]
{
	\paragraph{Пример.\\}
	{\tt
	\begin{tabular}{|p{0.4\linewidth}|p{0.4\linewidth}|}
	\hline
	Ввод 	& Вывод  	\\
	\hline
	#1 		& #2 		\\
	\hline
	#3		& #4		\\
	\hline
	#5		& #6		\\
	\hline
	\end{tabular}
	}
}

\newcommand{\exampleeee}[8]
{
	\paragraph{Пример.\\}
	{\tt
	\begin{tabular}{|p{0.4\linewidth}|p{0.4\linewidth}|}
	\hline
	Ввод 	& Вывод  	\\
	\hline
	#1 		& #2 		\\
	\hline
	#3		& #4		\\
	\hline
	#5		& #6		\\
	\hline
	#7		& #8		\\
	\hline
	\end{tabular}
	}
}

\newcommand{\exampleeeee}[5]
{
	\paragraph{Пример.\\}
	{\tt
	\begin{tabular}{|p{0.4\linewidth}|p{0.4\linewidth}|}
	\hline
	Ввод 	& Вывод  	\\
	\hline
	#1		\\
	\hline
	#2		\\
	\hline
	#3		\\
	\hline
	#4		\\
	\hline
	#5		\\
	\hline
	\end{tabular}
	}
}

\newcommand{\examplepic}[3]
{
	\subsection*{Пример.}
	{\tt
	\noindent
	\begin{tabular}{|p{0.1\linewidth}|p{0.1\linewidth}|p{0.5\linewidth}|}
	\hline
	Ввод 	& Вывод  	& Пояснение\\
	\hline
	#1 		& #2 		& #3\\
	\hline
	\end{tabular}
	}
}


\newcommand{\excomm}[1]
{
	\paragraph{Комментарий. \\}
	\textit{#1}
}

\begin{document}

\head{Открытая олимпиада по программированию \\ Весенний тур 2015}{18 марта 2015}{Казахстанский филиал МГУ имени М.В.Ломоносова}{г.~Астана}

\subsection*{A. Аdamant digit}

Надира с удивлением заметила, что её новая ручка странным образом не способна писать числа, в которых есть хотя бы две различные цифры (например, 34 или 511). Надира сначала хотела расстроиться, но потом ее внезапно осенило! На самом деле осталось ещё бесконечно много натуральных чисел, которые она может выписать: 1, 2, 3, 4, 5, 6, 7, 8, 9, 11, 22, 33, $\dots$. Интересно, каково же $N$-е число в этой последовательности?

\informat{Одно целое положительное число $N$ (от 1 до $10^4$).}

\outformat{Одно целое положительное число --- $N$-е число последовательности.}

\examplee{20}{222}{100}{111111111111}


\subsection*{B. Binary palindromes}

Мирас изобрёл машину времени и теперь за символическую плату отправляет всех желающих в ближайшее прошлое. Вперёд она не работает, да и кому это нужно? Вот и сейчас к Мирасу посетитель, который хочет вернуть старые добрые деньки, когда трава была зеленее, люди --- добрее, доллар --- дешевле, а двоичная запись номера года --- палиндромом (читалась одинаково и слева направо, и справа налево). Правда, любитель поностальгировать сам не знает, какой год он имеет в виду, поэтому решать придётся Мирасу. Какой ближайший год удовлетворяет всем требованиям и нужно ли вообще отправлять этого господина в прошлое? Вдруг сейчас за окном всё то, чего просит посетитель?

\informat{Одно целое положительное число $N$ (от 1 до $10^{18}$).}

\outformat{Одно целое положительное число --- максимальный двоичный палиндром, не превосходящий $N$ (в десятичной системе счисления).}

\examplee{32}{31}{2018}{2015}

\excomm{$2015 = 11111011111_2$ --- двоичный палиндром.}



\subsection*{C. Composition of matrices}

Андрей наказан за то, что он в попытках продлить зиму залил льдом коридор университета и с радостными криками катался по нему на коньках. Теперь он не может выйти из аудитории, пока не выполнит своё задание. Задача такова: на доске выписаны некоторые операции, которые могут быть последовательно применены к некоторому двумерному вектору. Всего операций три вида: $R$ --- поворот на фиксированный угол $\alpha$, $X$ --- проецирование на ось $Ox$, $Y$ --- проецирование на ось $Oy$. Помогите Андрею и укажите минимальное целое положительное значение угла $\alpha$ в градусах, при котором существует такой вектор, что после применения к нему комбинации этих преобразований он не изменит свою длину.

\informat{Последовательность символов $R$, $X$, $Y$, оканчивающаяся символом $E$ (всего не более 1000 символов).}

\outformat{Одно целое положительное число $\alpha$ от 1 до 360 --- ответ в градусах (если ответа не существует, выведите -1).}

\examplee{XRRRXE}{60}{XRRRXRRRYE}{-1}

\excomm{В первом примере можно взять вектор $(1, 0)$. После проецирования на ось $Ox$, трёх поворотов на $60^{o}$ и повторного проецирования он сохранит свою единичную длину.}


\subsection*{D. Deep tree}

Наевшись до отвала торта, Илья решил поиграть во что-нибудь новое на своём компьютере. Его выбор пал на популярную инди--игру SuperMegaDeepTree. Её геймплей довольно прост: каждый раз, как он щёлкает по пузырьку, содержащему пару чисел $(a, b)$, на экране появляются две новых пары: $(a, a+b)$ и $(a+b, b)$. Какое минимальное число щелчков нужно сделать Илье, чтобы достичь пары $(p, q)$ и закончить игру, если в начале у него нет ничего, кроме одного пузырька с парой $(1, 1)$?

\informat{Два целых положительных числа $p$ и $q$ (от 1 до $10^{18}$).}

\outformat{Одно целое число -- минимальное число щелчков, после которого на экране появится пара $(p, q)$. Если эта цель недостижима, то вывести -1.}

\examplee{2 3}{2}{2015 2020}{-1}

\excomm{После щелчка на $(1, 1)$ появляются пары $(1, 2)$ и $(2, 1)$. В первом примере победа достигается после щелчка по $(2, 1)$, то есть всего нужно только два действия.}



\subsection*{E. Empty сornet}

Вадим хочет купить Маше мороженого и ради этого даже вышел на улицу и нашёл лавку, где это самое мороженое и продаётся. Однако продавец оказался с довольно специфичными вкусами. Чтобы заказать в его ларьке мороженого, нужно назвать ему три точки трёхмерного пространства, а с ними продавец делает следующее: он проводит три отрезка, соединяющих каждую из этих точек с началом координат $(0, 0, 0)$, оборачивает полученную конструкцию в кулёк и уже в него наливает мороженое до самых краёв. Вадим, не особо подумав, тут же назвал точки $(x_1, y_1, z_1)$, $(x_2, y_2, z_2)$ и $(x_3, y_3, z_3)$. Чуть позже он осознал, что поспешил и начал волноваться, а какой же вообще объём мороженого получит Маша?

\informat{Три строки с координатами. В $i$-й строке 3 целых числа (от -1000 до 1000): $x_i$, $y_i$, $z_i$.}

\outformat{Одно вещественное число --- максимальный объём, который можно налить (с абсолютной погрешностью не более 0.001).}

\examplee{%
1 0 1 \newline
0 1 1 \newline
0 0 2}%
{0.166667}{%
1 0 1 \newline
0 1 1 \newline
0 0 -2}%
{0}

\excomm{Сила тяжести направлена вдоль вектора $(0, 0, -1)$.}



\subsection*{F. Fantastic system}

Денис в душе бунтарь и революционер и общественные устои он принимать никак не желает. В частности, он отказывается верить в то, что в двоичной записи можно использовать только две цифры: 0 и 1. Почему бы не использовать и цифру 2? Это ведь всё-таки ДВОИЧНАЯ запись. Вот только когда он выступил со своим предложением в своей комнате в общежитии (необходимый предварительный этап перед выступлением в Парламенте), Владислав резонно заметил, что в этом случае многие числа будут иметь несколько различных видов записи. Например, число 9 можно записать не только как 1001, но и как 121 и даже как 201. На что Денис парировал: <<Чем больше, тем лучше!>> А действительно, сколько же способов записать какое-то число в этой фантастической системе счисления?

\informat{Одно целое положительное число $N$ (от 1 до $10^6$).}

\outformat{Одно целое неотрицательное число --- количество способов записать число $N$ в фантастической системе счисления.}

\examplee{9}{3}{2015}{6}



\subsection*{G. Great graph}

Саша хотела бы увидеть на доске число $b$. Но просто написать это число она не может, так как связана ограничениями, которые она сама себе зачем-то придумала. По всей видимости, так она хочет развить в себе дух студента МГУ. Ограничения заключаются в следующем. К последнему написанному числу $p$ можно приписать число $q$, только если:
\begin{enumerate}
\item $q$ не превосходит $n$;
\item $p+q$ --- простое;
\item $q$ еще не встречалось на доске.
\end{enumerate}
В начале у Саши есть только число $a$. Найдите минимальное значение $n$, при котором эта задача выполнима.

\informat{Два целых положительных числа $a$ и $b$ (от 1 до 2000).}

\outformat{Одно целое положительное число $n$ (от 1 до 2000) --- минимальный из возможных ответов и -1, если такого n нет.}

\examplee{13 1}{4}{24 3}{5}

\excomm{Цепочка в первом примере такая: 13, 4, 1; во втором: 24, 5, 2, 1, 4, 3.}

\subsection*{H. H}

Ануар отлично знает английский алфавит, но это никак не поможет ему решить задачу.

\informat{Одно целое положительное число $n$ от 1 до 10.}

\example{8}{H}



\subsection*{I. Insidious time limit}

Тимур и Ерулан, готовясь к олимпиаде, нашли старую задачу. Путь к вершине спортивного программирования нелёгок, и пока у них не всё получается, что случилось и с этой задачей. Уже который раз подряд они получают вердикт TL на 20 тесте. Может, получится у вас? Вот сама проблема: найти $a_n$, где $a_k = a_{k-2} + a_{k-3}$ для всех $k$, начиная с 3.

\informat{Две строки: \newline на первой строке одно целое число $n$ (от 0 до $10^{18}$), на второй строке три целых числа $a_0$, $a_1$, $a_2$ (от 0 до $10^{18}$).}

\outformat{Одно целое неотрицательное число $a_n$ --- ответ по модулю $10^9$.}

\examplee{3 \newline 10 11 12}{21}{0 \newline 5 11 20}{5}



\subsection*{J. Jagged sequence}

По какой-то причине Асылжан наказан. Он сам не понимает, за что. Возможно, это как-то связано с непонятными криками в коридоре, которые он слышал сквозь дрёму, пока спал на лекции. В любом случае, теперь он не может выйти из аудитории, пока не выполнит своё задание. Задача такова: на доске выписана некоторая последовательность целых чисел. Ему нужно сделать из неё возрастающую арифметическую прогрессию с разностью 1, при этом за один ход он может прибавить единицу к какому-то числу или отнять единицу от какого-то одного числа. Понятное дело, Асылжан хочет избавиться от наказания как можно быстрее. Скажите, за какое минимальное число операций он справится?

\informat{Две строки: \newline в первой строке целое положительное $n$ (от 1 до 5000), во второй --- $n$ целых чисел $a_1$, $\dots$, $a_n$ (каждое от $-10^6$ до $10^6$).}

\outformat{Одно целое неотрицательное число --- минимальное количество операций.}

\examplee{4 \newline 1 7 5 6}{5}{3 \newline -2 -1 0}{0}

\excomm{Первый пример: прибавить 2 раза единицу к первому элементу, вычесть 3 раза единицу из второго.}

\end{document} 
