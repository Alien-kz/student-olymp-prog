\documentclass[11pt, a5paper]{article}

\usepackage[T2A]{fontenc}
\usepackage[utf8]{inputenc}
\usepackage[english, russian]{babel}
\usepackage{amssymb, amsfonts, amsmath}
\usepackage{mathtext}
\usepackage{comment}
\usepackage{graphicx}
\usepackage{listings}
\usepackage{geometry}
\usepackage{lscape}
\usepackage{wrapfig}
\usepackage[indentfirst, pagestyles, explicit]{titlesec}
\usepackage{longtable}

\usepackage{tikz}
\usetikzlibrary{patterns}		%draw pictures: fill
\usetikzlibrary{calc}			%draw pictures: coordinate calc
% \usetikzlibrary{external}		%draw pictures: cache pitcures
% \tikzexternalize				%cache pictures
\usepackage{listofitems}		%list of arguments (pictures)
\usepackage{enumitem}			%enumarate parameters

\geometry{left=1cm, right=1cm, top=2cm, bottom=2cm, twoside}

\titleformat{\section}{\Large\bfseries\center}{}{0pt}{#1}
\titleformat{\subsection}{\Large\bfseries\center}{}{0pt}{#1}

\newpagestyle{mystyle}{
	\headrule
	\sethead[\sectiontitle][][]
			{}{}{\subsectiontitle}
	\setfoot[\thepage][][]
			{}{}{\thepage}
}

\lstset{language=C++}
\lstset{basicstyle=\footnotesize\ttfamily} 
\lstset{keywordstyle=\color{blue}}
\lstset{frame=single}
\lstset{numbers=left}

\sloppy

\newcommand{\accept}[2]{
	\centerline{\boxed{#1}}
	\newline
	\centerline{\scriptsize{#2}}
}
\newcommand{\reject}[1]{
	\centerline{#1}
}

\newcommand{\informat}[1]
{
	\paragraph{Ввод.\\} #1
}

\newcommand{\outformat}[1]
{
	\paragraph{Вывод.\\} #1
}

\newcommand{\example}[2]
{
	\paragraph{Пример.\\}
	{\tt
	\begin{tabular}{|p{0.4\linewidth}|p{0.4\linewidth}|}
	\hline
	Ввод & Вывод \\
	\hline
	#1 & #2		\\
	\hline
	\end{tabular}
	}
}

\newcommand{\examplee}[4]
{
	\paragraph{Пример.\\}
	{\tt
	\begin{tabular}{|p{0.4\linewidth}|p{0.4\linewidth}|}
	\hline
	Ввод 	& Вывод  	\\
	\hline
	#1 		& #2 		\\
	\hline
	#3		& #4		\\
	\hline
	\end{tabular}
	}
}

\newcommand{\examplEEE}[6]
{
	\paragraph{Пример.\\}
	{\tt
	\begin{tabular}{|p{0.5\linewidth}|p{0.3\linewidth}|}
	\hline
	Ввод 	& Вывод  	\\
	\hline
	#1 		& #2 		\\
	\hline
	#3		& #4		\\
	\hline
	#5		& #6		\\
	\hline
	\end{tabular}
	}
}

\newcommand{\exampleee}[6]
{
	\paragraph{Пример.\\}
	{\tt
	\begin{tabular}{|p{0.4\linewidth}|p{0.4\linewidth}|}
	\hline
	Ввод 	& Вывод  	\\
	\hline
	#1 		& #2 		\\
	\hline
	#3		& #4		\\
	\hline
	#5		& #6		\\
	\hline
	\end{tabular}
	}
}

\newcommand{\exampleeee}[8]
{
	\paragraph{Пример.\\}
	{\tt
	\begin{tabular}{|p{0.4\linewidth}|p{0.4\linewidth}|}
	\hline
	Ввод 	& Вывод  	\\
	\hline
	#1 		& #2 		\\
	\hline
	#3		& #4		\\
	\hline
	#5		& #6		\\
	\hline
	#7		& #8		\\
	\hline
	\end{tabular}
	}
}

\newcommand{\exampleeeee}[5]
{
	\paragraph{Пример.\\}
	{\tt
	\begin{tabular}{|p{0.4\linewidth}|p{0.4\linewidth}|}
	\hline
	Ввод 	& Вывод  	\\
	\hline
	#1		\\
	\hline
	#2		\\
	\hline
	#3		\\
	\hline
	#4		\\
	\hline
	#5		\\
	\hline
	\end{tabular}
	}
}

\newcommand{\examplepic}[3]
{
	\subsection*{Пример.}
	{\tt
	\noindent
	\begin{tabular}{|p{0.1\linewidth}|p{0.1\linewidth}|p{0.5\linewidth}|}
	\hline
	Ввод 	& Вывод  	& Пояснение\\
	\hline
	#1 		& #2 		& #3\\
	\hline
	\end{tabular}
	}
}


\newcommand{\excomm}[1]
{
	\paragraph{Комментарий. \\}
	\textit{#1}
}

\renewcommand{\tabcolsep}{0.1cm}
\def\arraystretch{2}

\newcommand{\result}[2]
{
	\begin{landscape}
	\begin{center}

	\begin{large}
		Результаты\\
		#2\\
		\vspace{0.5cm}
	\end{large}	
	
	\begin{tiny}
	\input{results/#1}
	\end{tiny}
	\end{center}
	\end{landscape}
	\newpage
}
\newcommand{\resultind}[2]
{
	\newpage
	\begin{center}

	\begin{large}
		Результаты\\
		#2\\
		\vspace{0.5cm}
	\end{large}	
	
	\begin{tiny}
	\input{results/#1}
	\end{tiny}
	\end{center}
	\newpage
}

\newcommand{\problemauthor}[1]{
\begin{flushright}
\textit{Автор: #1}
\end{flushright}
}

\newcommand{\problemofferer}[1]{
\begin{flushright}
\textit{Предложил: #1}
\end{flushright}
}

\newcommand{\head}[2]
{
    \subsection{#1}
    \begin{center}
	#2
	\end{center}
}


\begin{document}

\begin{titlepage}
\begin{center}
\vfill

Московский государственный университет имени М.В.~Ломоносова\\
Казахстанский филиал\\

\vfill

\begin{Large}
\textbf{Студенческие олимпиады по программированию\\
Казахстанского филиала МГУ. \\
Задачи и указания.\\
2013--2018 гг.\\
}
\end{Large}

\vfill

Астана\\
2018

\end{center}
\end{titlepage}

\setcounter{page}{2}


\thispagestyle{empty}

\noindent{\bf УДК} \\
{\bf ББК}\\
{\bf Л}
\vspace{0.7 cm}

{\bf Рецензент:}

кандидат физико--математических наук, доцент кафедры математики и информатики Казахстанского филиала МГУ имени М.~В.~Ломоносова Нетесов~В.~В.

\vspace{0.5 cm}

{\bf Авторы:}

преподаватели Казахстанского филиал МГУ имени М.~В.~Ломоносова Абдикалыков~А.~К., Баев~А.~Ж.

\vspace{0.5 cm}

{\bf TeX--верстка:}

Баев~А.~Ж.

\vspace{0.5 cm}

В настоящем сборнике представлены 113 задач с указаниями, которые предлагались на 13 студенческих олимпиадах по программированию Казахстанского филиала МГУ имени М.В.Ломоносова за период с 2013 по 2018 год. 

Сборник  адресован  всем интересующимся олимпиадным движением.

\vspace{0.5 cm}


{\bf ISBN}  

\newpage
\section{Предисловие}

Традиция ежегодных студенческих олимпиад Казахстанского филиала МГУ по программированию зародилась со дня открытия филиала в 2001 году. Олимпиады проводились один раз в год в декабре при непосредственном участии доцента кафедры математики и информатики Казахстанского филиала МГУ Нетесова Виктора Викторовича и удаленном участии доцента кафедры системного программирования факультета ВМК Чернова Александра Владимировича, который является автором известной системы для проведения олимпиад по программированию ejudge. 

С 2013-2014 учебного года олимпиада разделилась на весенний и осенний тур с командным участием, а в 2015-2016 учебном году добавился индивидуальный тур с индивидуальным участием. С 2015 года олимпиада проводится с онлайн--версией, что позволяет принять участие в олимпиаде участникам из других городов, в частности студентам 3--4 курса Казахстанского филиала МГУ, которые обучаются в Москве. За последние годы в олимпиаде принимали участие студенты учебных заведений Казахстана (Астана, Алматы) и городов других стран (Москва, Ташкент).

В данном сборнике содержится 113 задач, которые были предложены на девяти олимпиадах филиала МГУ в период с 2013 по 2018 год. Значительная часть задач была разработана и подготовлена преподавателями Казахстанского филиала МГУ имени М.В.~Ломоносова Абдикалыковым Абдикожой Кожанасиридиновичем и Баевым Аленом Жуматаевичем. 

Активное участие в подготовке олимпиад принимали и самми студенты. Так, большую помощь в организации олимпиады 2012 года оказал студент четвертого курса ВМК Максимец Илья. Зимний тур 2015 года подготовили студенты второго курса: Камалбеков Тимур, Журавская Александра, Абайулы Ерулан и Жусупов Али. А зимний тур 2017 года: Аскергалиев Ануар, Бекмаганбетов Бекарыс и Шарипов Азат.

Авторы выражают благодарность всем студентам, кто принимал участие в олимпиадах филиала.

\begin{flushright}
Директор Казахстанского филиала

А.В.Сидорович
\end{flushright}

\newpage

\pagestyle{mystyle}

\tableofcontents

\newpage

\section{Команды Филиала на чемпионате ACM~ICPC}

В текстах задач олимпиад Казахстанского филиала упоминаются студенты, которые активно участвовали в олимпиадном движении филиала в период с 2012 по 2017 год и, в частности, принимали участие в чемпионате мира по программированию ACM ICPC.

Чемпионат мира по программированию проводится в несколько этапов. Четвертьфинал чемпионата мира для команд из Казахстана проводится параллельно в Алматы и Астане в конце октября. Обычно он собирает около 100 команд, представляющих около 30 университетов Казахстана. Около 30 лучших команд (но не более 4 команд из одного университета) приглашаются на следующий этап --- полуфинал чемпионата мира.

Команды, отобранные на 16 четвертьфиналах Северо-Восточного Европейского региона, встречаются на полуфинале в начале декабря. Соревнование проводится параллельно на одном и том же наборе задач сразу на четырех площадках: Ташкент (до 2015 года), Алматы (с 2016 года), Барнаул, Тбилиси и Санкт-Петербург. Как правило, это около 240 команд, представляющие 120 ВУЗов данного региона. Около 15 лучших команд (но не более 1 команды из одного университета) приглашаются на финал.

Всего за 6 сезонов с 2012 по 2017 год в четвертьфинале команды представляли филиал 21 раз, из которых 17 раз успешно квалифицировались на полуфинал. Команда филиала принимала участие в полуфинале на трех разных площадках: г. Ташкент в 2012 году, г. Барнаул в 2013 и 2015 годах и г. Алматы в 2016 и в 2017 годах. Лучшим достижением команд филиала на полуфинале является дипломы Средней Азии (г. Ташкент, 2012), Сибири (г. Барнаул, 2015), Северо-Восточного Европейского региона (г. Алматы, 2016).

Команды филиала состоят из студентов, обучающихся по направлениям <<Математика>> и <<Прикладная математика>>. Многие участники команд филиала знакомятся с олимпиадами по програмированию в течении первого года обучения. Уже ко второму курсу они часто выступают на равных с опытными соперниками из других ВУЗов Казахстана.

Условия задач, результаты и материалы четвертьфиналов и полуфинала чемпионата мира по программированию ACM ICPC доступны на официальном сайте организатора в лице Санкт-Петербургского национального исследовательского университета информационных технологий, механики и оптики: https://neerc.ifmo.ru.

\subsubsection*{2012--2013 учебный год}

\paragraph{Четвертьфинал чемпионата (Казахстан).} Казахстанский филиал МГУ занял 6 место из 28 ВУЗов Казахстана, а лучшая команда филиала --- 22 место из 110 команд.

\paragraph{Полуфинал чемпионата (Средняя Азии).} Казахстанский филиал МГУ занял 11 место из 18 ВУЗов Средней Азии, а лучшая команда филиала --- 22 место из 47 команд Средней Азии (15 место из 29 среди команд из Казахстана). Данный результат отмечен \textbf{дипломом третьей степени среди команд Казахстана}.

\paragraph{Полуфинал чемпионата (СНГ).} Казахстанский филиал МГУ занял 92 место из 129 ВУЗов СНГ, а лучшая команда филиала --- 153 место из 229 команд. Данный результат отмечен \textbf{дипломом третьей степени среди команд Средней Азии}. К слову, команды МГУ заняли 2, 6, 7, 9 и 10 места.

\begin{center}
\begin{tabular}{|p{1.8cm}|p{5.5cm}|p{1.5cm}|p{1.6cm}|l|}
\hline
Команда & Состав & 1/4 \newline Астана & 1/2 \newline Ташкент\\
\hline
Jusual &
Максимец Илья, ВМ-4, \newline
Суворова Юлия, ММ-2, \newline
Жадиков Дамир, ММ-1. 
&
22 место \newline
4 задачи
&
22 место \newline
2 задачи
\\
\hline
\end{tabular}
\end{center}

\newpage
\begin{center}
\includegraphics[width=0.7\linewidth]{diploma/2012-astana}

Диплом 3 степени среди команд Казахстана 2012\\
Команда Jusual (Максимец, Суворова, Жадиков).
\end{center}

\begin{center}
\includegraphics[width=0.7\linewidth]{diploma/2012-tashkent}

Диплом 3 степени среди команд Центральной Азии 2012\\
Команда Jusual (Максимец, Суворова, Жадиков).
\end{center}
\newpage

\subsubsection*{2013--2014 учебный год}

\paragraph{Четвертьфинал чемпионата (Казахстан).} Казахстанский филиал МГУ занял 10 место из 40 ВУЗов Казахстана, а лучшая команда филиала --- 29 место из 91 команды. Данный результат отмечен \textbf{дипломом в номинации <<Лучшая гостевая команда на площадке в Астане>>}.

\paragraph{Полуфинал чемпионата (Сибирь).} Казахстанский филиал МГУ занял 17 место из 29 ВУЗов Сибири, а лучшая команда филиала --- 25 место из 44 команд Сибири (17 место из 30 среди команд из Казахстана).

\paragraph{Полуфинал чемпионата (СНГ).} Казахстанский филиал МГУ занял 78 место из 117 ВУЗов СНГ, а лучшая команда филиала --- 147 место из 228 команд. К слову, команды МГУ заняли 2, 10, 11, 12, 27 места.

\begin{center}
\begin{tabular}{|p{1.8cm}|p{5.5cm}|p{1.5cm}|p{1.6cm}|l|}
\hline
Команда & Состав & 1/4 \newline Астана & 1/2 \newline Барнаул \\
\hline
Big dipper &
Тлеубаев Адиль, ВМ-2, \newline
Таранов Денис, ВМ-1, \newline
Шокетаева Надира, ММ-1. 
&
29 место \newline
5 задач
&
25 место \newline
2 задачи
\\
\hline
Lord \newline Bendtner \newline Team &
Седякин Илья, ВМ-1, \newline
Таскынов Ануар, ВМ-1, \newline
Васильев Андрей, ВМ-1. 
&
46 место \newline
4 задачи
&
-
\\
\hline
\end{tabular}
\end{center}

\newpage
\mbox{}
\vfill
\begin{center}
\includegraphics[width=0.8\linewidth]{diploma/2013-astana}

Диплом в номинации <<Лучшая гостевая команда на площадке в Астане>>

Команда Big dipper (Тлеубаев, Таранов, Шокетаева).
\end{center}
\vfill
\mbox{}
\newpage
\subsubsection*{2014--2015 учебный год}

\paragraph{Четвертьфинал чемпионата (Казахстан).} Казахстанский филиал МГУ занял 8 место из 30 ВУЗов Казахстана, а лучшая команда филиала --- 24 место из 98 команды.

\begin{center}
\begin{tabular}{|p{1.8cm}|p{5.5cm}|p{1.5cm}|p{1.6cm}|l|}
\hline
Команда & Состав & 1/4 \newline Астана & 1/2 \newline Барнаул\\
\hline
Lord \newline Bendtner \newline Team &
Седякин Илья, ВМ-2, \newline
Таскынов Ануар, ВМ-2, \newline
Вержбицкий Владислав, ВМ-2. 
&
24 место \newline
4 задачи
&
-
\\
\hline
MSU 4 &
Амир Мирас, ВМ-2 \newline
Шабхатов Асылжан, ВМ-2 \newline
Токтаганов Адиль, ММ-2 &
39 место \newline
3 задачи
&
-
\\
\hline
Snowy \newline Cube &
Журавская Александра, ВМ-1 \newline
Камалбеков Тимур, ВМ-1 \newline
Абайулы Ерулан, ВМ-1
&
42 место \newline
2 задачи
&
x
\\
\hline
Big \newline Dipper &
Таранов Денис, ВМ-2 \newline
Шокетаева Надира, ММ-2 \newline
Жусупов Али, ММ-1
&
44 место \newline
2 задачи
&
x
\\
\hline
\end{tabular}
\end{center}

\newpage

\subsubsection*{2015--2016 учебный год}

\paragraph{Четвертьфинал чемпионата (Казахстан).} Казахстанский филиал МГУ занял 6 место из 15 ВУЗов Казахстана, а лучшая команда филиала --- 23 место из 78 команд.

\paragraph{Полуфинал чемпионата (Сибирь).} Казахстанский филиал МГУ занял 12 место из 29 ВУЗов Сибири, а лучшая команда филиала --- 17 место из 44 команд Сибири (14 место из 16 среди команд из Казахстана).

\paragraph{Полуфинал чемпионата (СНГ).} Казахстанский филиал МГУ занял 70 место из 122 ВУЗов СНГ, а лучшая команда филиала --- 129 место из 224 команд. Данный результат отмечен дипломом третьей степени среди команд Сибири. К слову, команды МГУ заняли 4, 8, 65 и 68 места.

\begin{center}
\begin{tabular}{|p{1.8cm}|p{5.5cm}|p{1.5cm}|p{1.6cm}|l|}
\hline
Команда & Состав & 1/4 \newline Астана & 1/2 \newline Барнаул\\
\hline
Snowy \newline Cube &
Журавская Александра, ВМ-2 \newline
Камалбеков Тимур, ВМ-2 \newline
Абайулы Ерулан, ВМ-2
&
23 место \newline
6 задач
&
17 место \newline
3 задачи
\\
\hline
Big \newline Dipper &
Тлеубаев Адиль, ВМ-4, \newline
Таранов Денис, ВМ-3, \newline
Шокетаева Надира, ММ-3. 
&
25 место \newline
6 задач
&
-
\\
\hline
Lord \newline Bendtner \newline Team &
Седякин Илья, ВМ-3, \newline
Таскынов Ануар, ВМ-3, \newline
Вержбицкий Владислав, ВМ-3 
&
29 место \newline
5 задач
&
-
\\
\hline
Die \newline Perdimus &
Жусупов Али, ММ-2 \newline 
Турганбаев Сатбек, ВМ-2 \newline
Омаров Темирхан, ВМ-2
&
37 место \newline
4 задачи
&
-
\\
\hline
\end{tabular}
\end{center}

\newpage
\mbox{}
\vfill
\begin{center}
\includegraphics[width=0.8\linewidth]{diploma/2015-barnaul}

Диплом 3 степени среди команд Сибири 2015

Команда Snowy Cube (Журавская, Камалбеков, Абайулы).
\end{center}
\vfill
\mbox{}
\newpage



\subsubsection*{2016--2017 учебный год}

\paragraph{Четвертьфинал чемпионата (Казахстан).} Казахстанский филиал МГУ занял 7 место из 29 ВУЗов Казахстана, а лучшая команда филиала --- 20 место из 141 команды.

\paragraph{Полуфинал чемпионата (Средняя Азия).} Казахстанский филиал МГУ занял 5 место из 16 ВУЗов Средней Азии, а лучшая команда филиала --- 9 место из 42 команд Средней Азии (8 место из 28 среди команд из Казахстана). 

\paragraph{Полуфинал чемпионата (СНГ).} Казахстанский филиал МГУ занял 49 место из 109 ВУЗов СНГ, а лучшая команда филиала --- 93 место из 228 команд. Данный результат отмечен \textbf{дипломом третьей степени среди команд СНГ}. К слову, команды МГУ заняли 25, 27 и 72 места.

\begin{center}
\begin{tabular}{|p{2.0cm}|p{5.8cm}|p{1.5cm}|p{1.6cm}|l|}
\hline
Команда & Состав & 1/4 \newline Астана & 1/2 \newline Алматы\\
\hline
Snowy \newline Cube &
Журавская Александра, ВМ-3 \newline
Камалбеков Тимур, ВМ-3 \newline
Абайулы Ерулан, ВМ-3
&
20 место \newline
4 задачи
&
9 место \newline
3 задачи
\\
\hline
Big \newline Dipper &
Тлеубаев Адиль, ВМ-м, \newline
Амир Мирас, ВМ-4, \newline
Шокетаева Надира, ММ-4. 
&
38 место \newline
3 задачи
&
-
\\
\hline
Lord \newline Bendtner \newline Team &
Седякин Илья, ВМ-4, \newline
Таскынов Ануар, ВМ-4, \newline
Вержбицкий Владислав, ВМ-4
&
51 место \newline
3 задачи
&
-
\\
\hline
Nerzhul &
Болотников Димитрий, ММ-2 \newline
Газизов Куат, ММ-2 \newline
Аскергали Ануар, ВМ-1
&
52 место \newline
3 задачи
&
-
\\
\hline
Esprit &
Сеилов Айтмухаммед, ММ-2 \newline
Шарипов Азат, ВМ-1 \newline
Танкибаев Салима, ВМ-1
&
53 место \newline
3 задачи
&
x
\\
\hline
Complicate &
Коробов Павел, ММ-2 \newline
Бекмаганбектов Бекарыс, ММ-1 \newline
Кунакбаев Рамазан, ММ-1
&
69 место \newline
2 задачи
&
x
\\
\hline
\end{tabular}
\end{center}

\newpage
\mbox{}
\vfill
\begin{center}
\includegraphics[width=0.9\linewidth]{diploma/2016-almaty}

Диплом 3 степени среди команд Северо-Восточного европейского региона (СНГ).

Команда Snowy Cube (Журавская, Камалбеков, Абайулы).
\end{center}
\vfill
\mbox{}
\newpage

\subsubsection*{2017--2018 учебный год}

\paragraph{Одна восьмая чемпионата (Москва).} Команда Snowy Cube участвовала в Московской ветке от МГУ. Заняла 40 место из 301 команды (11 место среди 29 команд МГУ). Данный результат отмечен \textbf{дипломом победителей одной восьмой финала в г. Москве}.

\paragraph{Четвертьфинал чемпионата (Москва).} Команда Snowy Cube заняла 48 место из 87 команды (11 место среди 12 команд МГУ).

\begin{center}
\begin{tabular}{|p{1.8cm}|p{5.8cm}|p{1.5cm}|p{1.6cm}|l|}
\hline
Команда & Состав & 1/8 \newline Москва & 1/4 \newline Москва \\
\hline
Brain \newline Burst &
Камалбеков Тимур, ВМК-4 \newline
Журавская Александра, ВМК-4 \newline
Абайулы Ерулан, ВМК-4 
&
40 место \newline
8 задач
&
48 место \newline
3 задачи
\\
\hline
\end{tabular}
\end{center}

\newpage
\mbox{}
\vfill
\begin{center}
\includegraphics[width=0.9\linewidth]{diploma/2017-moscow}

Диплом победителей среди команд Москвы.

Команда Snowy Cube (Журавская, Камалбеков, Абайулы).
\end{center}
\vfill
\mbox{}
\newpage

\paragraph{Четвертьфинал чемпионата (Казахстан).} Казахстанский филиал МГУ занял 6 место из 14 ВУЗов Казахстана, а лучшая команда филиала --- 14 место из 122 команды.

\paragraph{Полуфинал чемпионата (Средняя Азия).} Казахстанский филиал МГУ занял 7 место из 45 ВУЗов Средней Азии, а лучшая команда филиала --- 10 место из 54 команд Средней Азии (6 место из 34 среди команд из Казахстана).

\paragraph{Полуфинал чемпионата (СНГ).} Казахстанский филиал МГУ занял 63 место из 126 ВУЗов СНГ, а лучшая команда филиала --- 127 место из 266 команд. К слову, команды МГУ заняли 2, 23, 30 и 40 места.

\begin{center}
\begin{tabular}{|p{2cm}|p{5.8cm}|p{1.5cm}|p{1.6cm}|l|}
\hline
Команда & Состав & 1/4 \newline Астана & 1/2 \newline Алматы\\
\hline
Brain \newline Burst &
Аскергали Ануар, ВМК-2 \newline
Бекмаганбетов Бекарыс, ММ-2 \newline
Шарипов Азат, ВМК-2 \newline
&
14 место \newline
7 задач
&
10 место \newline
3 задачи
\\
\hline
MSU 2 &
Кунакбаев Рамазан, ММ-2 \newline
Макатова Батима, ММ-2 \newline
Ержанов Жалгас, ВМК-2
&
25 место \newline
5 задач
&
-
\\
\hline
Murmaider &
Вагнер Алан, ВМК-1 \newline
Азатов Таир, ВМК-1 \newline
Понамарев Валерий, ВМК-1
&
30 место \newline
4 задачи
&
-
\\
\hline
Witty \newline name &
Газизов Куат, ММ-3 \newline
Болотников Димитрий, ММ-3 \newline
Коробов Павел, ММ-3 \newline
&
32 место \newline
4 задачи
&
-
\\
\hline
\end{tabular}
\end{center}

\newpage
\mbox{}
\vfill
\begin{center}
\includegraphics[width=0.9\linewidth]{diploma/2017-almaty}

Похвальная грамота среди команд Северо-Восточного европейского региона (СНГ).

Команда Brain Burst (Аскергалиев, Бекмаганбетов, Шарипов).
\end{center}
\vfill
\mbox{}
\newpage



\newpage

\section{Условия задач и результаты}

Во всех задачах до 2017 года были использованы ограничения по времени в 1 секунду, по памяти в 64 Мб. На задачи с 2018 года ограничения по времени в 1 секунду, по памяти в 512 Мб. На олимпиадах всегда гарантировалось наличие решения на языках pascal, C и С++, которые укладываются в данные ограничения. 

\makeatletter
\def\input@path{{problems/}}
\makeatother

\head{Зимний тур 2013}{11 декабря 2013}
\input{problems/2013-12}
\result{2013-12}{11 декабря 2013}

\head{Весенний тур 2014}{17 марта 2014}
\begin{center}
\begin{longtable}{|c|p{0.2\linewidth}|p{0.2\linewidth}|*{8}{p{0.025\linewidth}|}c|c|}
\hline 
№ & Команда & Состав & A & B & C & D & E & F & G & H & Итог & Штраф \\
\hline
\endhead
1 & Big Dipper &	Шокетаева Надира (ММ-11) 	\newline Таранов Денис (ВМ-11)		\newline Тлеубаев Адиль (ВМ-21) &
\accept{+}{0:09}  &
\accept{+}{0:12}  &
\accept{+}{2:33}  &
\accept{+2}{0:55}  &
\accept{+10}{2:07}  &
  &
  &
  &
5 &
596
 \\
\hline 
2 & msu\_25		&	Автайкина Мария (ВМ-21)		\newline Журавлев Вадим (ВМ-21)	\newline Овчинников Дмитрий (ВМ-21) &
\accept{+4}{1:11}  &
\accept{+}{0:12}  &
\accept{+1}{2:49}  &
  &
\accept{+}{2:01}  &
\reject{-3} &
  &
  &
4 &
473 \\
\hline 
3 & CMC AID		&	Оспанов Аят (ВМ-21)			\newline Ламонов Иван (ВМ-21)		\newline Солтанова Дана (ВМ-21) &
\accept{+1}{0:36}  &
\accept{+1}{0:09}  &
\accept{+}{1:33}  &
\reject{-1} &
  &
  &
  &
  &
3 &
178 \\
\hline 
4 & msu\_23		&	Седякин Илья (ВМ-11) \newline Васильев Андрей (ВМ-11)		\newline Таскынов Ануар (ВМ-11) &
\accept{+1}{0:08}  &
\accept{+}{0:10}  &
\accept{+1}{2:21}  &
  &
\reject{-2} &
  &
  &
  &
3 &
199 \\
\hline
5 & VM		&	Амир Мирас (ВМ-11)		\newline Матвеева Виктория (ВМ-11) &
\reject{-4} &
\accept{+1}{0:25}  &
\reject{-4} &
  &
  &
  &
  &
  &
1 &
45 \\
\hline
 & & Успешных попыток &
4  &
5  &
4  &
1  &
2  &
0  &
0  &
0  &
16  &
  \\
\hline 
 & & Всего попыток &
14  &
7  &
10  &
4  &
14  &
3  &
0  &
0  &
52  &
  \\
\hline 
\end{longtable} 
\end{center}
\result{2014-03}{17 марта 2014}

\head{Зимний тур 2014}{9 декабря 2014}
\input{problems/2014-12}
\result{2014-12}{9 декабря 2014}

\head{Весенний тур 2015}{18 марта 2015}
\input{problems/2015-03}
\result{2015-03}{18 марта 2015}

\head{Осенний тур 2015}{20 октября 2015}
\begin{center}
\begin{longtable}{|c|p{0.2\linewidth}|p{0.2\linewidth}|*{9}{p{0.025\linewidth}|}c|c|}
\hline 
№ & Команда & Состав & A & B & C & D & E & F & G & H & I & Итог & Штраф\\
\hline
\endhead
1 & Snowy Cube	 & Камалбеков Тимур (ВМ-2) \newline Абайулы Ерулан (ВМ-2) & 
\accept{+2}{0:27}  &
\accept{+}{0:10}  &
\accept{+}{0:47}  &
\accept{+1}{2:44}  &
\accept{+2}{1:11}  &
\accept{+}{0:44}  &
\accept{+2}{1:23}  &
\accept{+}{1:31}  &
  &
8 &
677\\
\hline
2 & Банка огурцов	 & Турганбаев Сатбек (ВМ-2) \newline Омаров Темирхан (ВМ-2) \newline Жусупов Али (ММ-2) & 
\accept{+2}{2:11}  &
\accept{+6}{1:57}  &
\accept{+1}{1:09}  &
\reject{-3} &
  &
\accept{+3}{2:17}  &
\accept{+1}{2:33}  &
\accept{+1}{2:46}  &
  &
6 &
1053\\
\hline
3 & School Dominators	 & Сёмочкин Константин (школа) \newline Ещанов Данияр (школа) \newline Сапаргалиев Олжас (школа) & 
\accept{+1}{0:32}  &
\reject{-4} &
\reject{-2} &
  &
\accept{+}{1:20}  &
\accept{+}{1:46}  &
  &
\reject{-5} &
  &
3 &
238\\
\hline
4 & Pascal/C++	 & Абдыкалик Чингиз (ВМ-1) \newline Кожемяк Виталий (ВМ-1) \newline Малик Алмат (ММ-2) & 
\accept{+}{0:32}  &
\reject{-7} &
\reject{-2} &
  &
  &
  &
\reject{-5} &
  &
  &
1 &
32\\
\hline
5 & HAWK	 & Керопян Агаси (ВМ-1) \newline Ахметов Дастан (ВМ-1) \newline Мырзабеков Руслан (ВМ-1) & 
\accept{+1}{1:45}  &
\reject{-8} &
  &
  &
  &
  &
  &
  &
  &
1 &
125\\
\hline
6 & Avadakedavra	 & Болотников Дмитрий (ММ-1) \newline Савицкая Нина (ММ-1) \newline Коробов Павел (ММ-1) & 
\accept{+}{2:45}  &
\reject{-5} &
  &
  &
  &
  &
  &
  &
  &
1 &
165\\
\hline
 & & Успешных попыток &
6  &
2  &
2  &
1  &
2  &
3  &
2  &
2  &
0  &
20 & \\
\hline 
 & & Всего попыток &
12  &
34  &
8  &
5  &
4  &
6  &
14  &
8  &
0  &
91 & \\
\hline 
\end{longtable} 
\end{center}
\result{2015-10}{20 октября 2015}

\head{Зимний тур 2015}{18 декабря 2015}
\input{problems/2015-12}
\resultind{2015-12}{18 декабря 2015}

\head{Весенний тур 2016}{30 апреля 2016}
\input{problems/2016-04}
\result{2016-04}{30 апреля 2016}

\head{Осенний тур 2016}{15 октября 2016}
\input{problems/2016-10}
\result{2016-10}{15 октября 2016}

\head{Зимний тур 2016}{13 декабря 2016}
\subsubsection*{A. A train problem}

\problemauthor{ Баев А.Ж.}

Ответ: $2 k (k-1) + 2 (n-k) (n-k+1)$.

Асимптотика: $O(1)$.



\subsubsection*{B. Bead garland}

\problemauthor{ Баев А.Ж.}

Количество различных гирлянд изначально равно $a_1 a_2 ... a_n$. Гирлянду длины один не выгодно объединять ни к какой другой гирлянде (вместо $a \cdot 1 < a + 1$), гирлянды большей длины наоборот выгоднее объединять между собой ($a_1 \cdot a_2 > a_1 + a_2$). Значит, выгоднее всего объединить все гирлянд, с длиной больше 1. При этом стоит обратить внимание на случай, когда имеются гирлянды только единичной длины.

Асимптотика: $O(n)$.



\subsubsection*{C. Champion}

\problemauthor{ Абдикалыков А.К.}

Вывод является буквами, ascii-код которых дан на вводе (32-й символ таблицы является пробел).

Асимптотика: $O(1)$.



\subsubsection*{D. Digits}

\problemauthor{ Абдикалыков А.К.}

Пусть $d[n][k]$ --- количество $n$-значных чисел, у которых сумма цифр равна $k$. Ясно, что если у числа отбросить последнюю цифру $z$, то получим число с суммой цифр $k - z$. Значит,
$$d[i][j] = \sum_{z=0}^{\min(9,j)} d[i-1][j-z].$$
Что легко просчитать от для всех $i$ от 1 до $n$ и $j$ от 1 до $k$. Начальные значения $d[0][0] = 1$ и $d[i][0] = 0$.

Асимптотика: $O(n k)$.



\subsubsection*{E. Elimination}

\problemauthor{ Баев А.Ж.}

Пусть $d[i][j]$ --- количество работающих участков у $i$-й жилы среди участков с $(j-m+1)$-го до $j$-го включительно, которые можно вычислить за $O(1)$ каждый:$$d[i][j] = d[i][j-1] + a[i][j] - a[i][j-m]$$
для всех $i$ от 1 до $k$ и $j$ от $m$ до $n$.
Ответом на задачу будет $\max_{m\leqslant j \leqslant n} c_j$,
где $c_j$ --- количество $d[i][j] = m$, для всех $i$ от 1 до $k$.

Асимптотика: $O(n k)$.



\subsubsection*{F. Forts}

\problemauthor{ Баев А.Ж.}

\begin{center}
\definecolor{light}{rgb}{0.85,0.85,0.85}
\begin{tikzpicture}[x=6, y=6]
\begin{scope}
  \clip (-10,-10) -- (-10,0) -- (0,0) -- (0, -10);
  \fill[fill=light, draw=none] (3, 4) circle (10);
\end{scope}
\draw (-7, 4) arc (180:270:10);
\draw (-10, 0) -- (0, 0);
\draw (0, -10) -- (0, 0);
\draw (3, 4) -- (0, 0);
\draw (3, 4) -- (-6.165, 0);
\draw (3, 4) -- (0, -5.539);
\node at (3, 4.5) {O};
\node at (-7, -1) {A};
\node at (-1, -7) {B};
\node at (-1, -1) {C};
\end{tikzpicture}
\end{center}

Искомая площадь равна нулю, если $a^2 + b^2 > r^2$. Иначе ее можно найти как разность площали кругового сектора $OAB$ и двух треугольников $AOC$ и $BOC$.

Асимптотика: $O(1)$.
\resultind{2016-12}{13 декабря 2016}

\head{Весенний тур 2017}{31 мая 2017}
\input{problems/2017-05}
\result{2017-05}{31 мая 2017}

\head{Осенний тур 2017}{28 октября 2017}
\input{problems/2017-10}
\result{2017-10}{28 октября 2017}

\head{Зимний тур 2017}{19 декабря 2017}
\input{problems/2017-12}
\resultind{2017-12}{19 декабря 2017}

\head{Весенний тур 2018}{19 мая 2018}
\input{problems/2018-05}
\result{2018-05}{19 мая 2018}

\newpage

\section{Указания}

\head{Зимний тур 2013}{11 декабря 2013}
\input{solutions/2013-12}

\head{Весенний тур 2014}{17 марта 2014}
\begin{center}
\begin{longtable}{|c|p{0.2\linewidth}|p{0.2\linewidth}|*{8}{p{0.025\linewidth}|}c|c|}
\hline 
№ & Команда & Состав & A & B & C & D & E & F & G & H & Итог & Штраф \\
\hline
\endhead
1 & Big Dipper &	Шокетаева Надира (ММ-11) 	\newline Таранов Денис (ВМ-11)		\newline Тлеубаев Адиль (ВМ-21) &
\accept{+}{0:09}  &
\accept{+}{0:12}  &
\accept{+}{2:33}  &
\accept{+2}{0:55}  &
\accept{+10}{2:07}  &
  &
  &
  &
5 &
596
 \\
\hline 
2 & msu\_25		&	Автайкина Мария (ВМ-21)		\newline Журавлев Вадим (ВМ-21)	\newline Овчинников Дмитрий (ВМ-21) &
\accept{+4}{1:11}  &
\accept{+}{0:12}  &
\accept{+1}{2:49}  &
  &
\accept{+}{2:01}  &
\reject{-3} &
  &
  &
4 &
473 \\
\hline 
3 & CMC AID		&	Оспанов Аят (ВМ-21)			\newline Ламонов Иван (ВМ-21)		\newline Солтанова Дана (ВМ-21) &
\accept{+1}{0:36}  &
\accept{+1}{0:09}  &
\accept{+}{1:33}  &
\reject{-1} &
  &
  &
  &
  &
3 &
178 \\
\hline 
4 & msu\_23		&	Седякин Илья (ВМ-11) \newline Васильев Андрей (ВМ-11)		\newline Таскынов Ануар (ВМ-11) &
\accept{+1}{0:08}  &
\accept{+}{0:10}  &
\accept{+1}{2:21}  &
  &
\reject{-2} &
  &
  &
  &
3 &
199 \\
\hline
5 & VM		&	Амир Мирас (ВМ-11)		\newline Матвеева Виктория (ВМ-11) &
\reject{-4} &
\accept{+1}{0:25}  &
\reject{-4} &
  &
  &
  &
  &
  &
1 &
45 \\
\hline
 & & Успешных попыток &
4  &
5  &
4  &
1  &
2  &
0  &
0  &
0  &
16  &
  \\
\hline 
 & & Всего попыток &
14  &
7  &
10  &
4  &
14  &
3  &
0  &
0  &
52  &
  \\
\hline 
\end{longtable} 
\end{center}

\head{Зимний тур 2014}{9 декабря 2014}
\input{solutions/2014-12}

\head{Весенний тур 2015}{18 марта 2015}
\input{solutions/2015-03}

\head{Осенний тур 2015}{20 октября 2015}
\subsubsection*{A. Alternative result} 

\problemauthor{Абдикалыков А.К.}

Несложно убедиться, что можно получить все значения от 0 до $3n$, кроме $3n-1$.

Асимптотика: $O(n)$.



\subsubsection*{B. Boolean} 

\problemauthor{Абдикалыков А.К.}

Необходимо было вывести $N$-е слово из текста.

Асимптотика: $O(1)$.



\subsubsection*{C. Car collection} 

\problemauthor{Баев А.Ж.}

Ответом на задачу является:
$$\sum_{i=1}^{n-1} \sum_{j=i+1}^{n} a_i a_j = \frac{1}{2} \left( (\sum_{i=1}^{n} {a_i})^2 - \sum_{i=1}^{n} a_i^2 \right).$$

Асимптотика: $O(n)$.

Замечание: наивное решение не проходит ограничения по времени.



\subsubsection*{D. Domino} 

\problemauthor{Баев А.Ж.}

Промоделируем падения слева направо и справа налево. Для этого найдем максимальную длину положительной подстроки массива $l_i$, где $l_i = a_i - a_{i-1} - h_{i-1}$, и максимальную длину положительной подстроки массива $r_i$, где $r_i = a_i - a_{i+1} - h_{i+1}$. Ответов будет максимум из первого и второго случая.

Асимптотика: $O(n)$.



\subsubsection*{E. Enlarged triangle} 

\problemauthor{Баев А.Ж.}

Пусть $S(a, b, c)$ --- площадь треугольника со сторонами $a$, $b$, $c$. Несложно проверить, что функция $f(m) = S(a + m, b + m, c + m)$ является монотонно возрастающей (при условии, что $m > 0$ и треугольник с данными сторонами существует). Значит, ответ можно найти бинарным поиском по $m$ на отрезке $[0; \sqrt{2 S}]$.

Асимптотика: $O(\log S)$.



\subsubsection*{F. Footprints} 

\problemauthor{Баев А.Ж.}

Обозначим начальную позицию (0, 0). Далее промоделируем шаги $(x_i, y_i)$. Минимальные размеры лабиринта будут $(\max\limits_{1 \leqslant i \leqslant n} x_i - \min\limits_{1 \leqslant i \leqslant n} x_i)$ и $(\max\limits_{1 \leqslant i \leqslant n} y_i - \min\limits_{1 \leqslant i \leqslant n} y_i)$ соответственно.

Асимптотика: $O(N)$.



\subsubsection*{G. Great divisors} 

\problemauthor{Абдикалыков А.К.}

Максимальный собственный делитель числа $n$ равен $n / p_n$, где $p_n$ --- минимальное простое число, на которое делится $n$. Последовательность $p_n$ легко построить, используя стандартный алгоритм решета Эратосфена (у всех еще не вычеркнутых чисел вида $p^2 + p \cdot j$ минимальным простым делителем будет $p$).

Асимптотика: $O(n \log n)$.



\subsubsection*{H. Honest gifts} 

\problemauthor{Баев А.Ж.}

Ясно, что максимальным количество наборов с общим количество $p$ синих и $q$ красных карандашей будет $(p, q)$ --- наибольший общий делитель $p$ и $q$. Поэтому достаточно перебрать все числа $i$ от 0 до $k$ и выбрать максимум из $gcd(a - i, b - (k - i))$.

Асимптотика: $O(k \log \max(a, b))$.



\subsubsection*{I. Inner subset} 

\problemauthor{Баев А.Ж.}

Необходимо посчитать количество способов выбрать подпоследовательность так, чтобы сумма чисел была кратна $k$. Обозначим $d[i][r]$ --- количество подпоследовательностей из первых $i$ элементов, которые в сумму дают остаток $r$ при делении на $k$. Каждое такое подножество можно получить, либо добавив $a[i]$ элемент к подножествами множества из первых $i-1$ с остатком суммы равным $(r - a[i]) \bmod k$, либо не добавляя $a[i]$ элемент:
$$d[i][r] = d[i-1][r] + d[i-1][(r - a[i]) \bmod k].$$
Инициализировать динамику можно $d[0][0] = 1$ и $d[0][r] = 0$ при $r$ от 1 до $k-1$.

Асимптотика: $O(n k)$.

Замечание: не стоит забывать производить каждое сложение по модулю $10^9 + 7$, иначе произойдет переполнение ответа.

\head{Зимний тур 2015}{18 декабря 2015}
\input{solutions/2015-12}

\head{Весенний тур 2016}{30 апреля 2016}
\input{solutions/2016-04}

\head{Осенний тур 2016}{15 октября 2016}
\input{solutions/2016-10}

\head{Зимний тур 2016}{13 декабря 2016}
\subsubsection*{A. A train problem}

\problemauthor{ Баев А.Ж.}

Ответ: $2 k (k-1) + 2 (n-k) (n-k+1)$.

Асимптотика: $O(1)$.



\subsubsection*{B. Bead garland}

\problemauthor{ Баев А.Ж.}

Количество различных гирлянд изначально равно $a_1 a_2 ... a_n$. Гирлянду длины один не выгодно объединять ни к какой другой гирлянде (вместо $a \cdot 1 < a + 1$), гирлянды большей длины наоборот выгоднее объединять между собой ($a_1 \cdot a_2 > a_1 + a_2$). Значит, выгоднее всего объединить все гирлянд, с длиной больше 1. При этом стоит обратить внимание на случай, когда имеются гирлянды только единичной длины.

Асимптотика: $O(n)$.



\subsubsection*{C. Champion}

\problemauthor{ Абдикалыков А.К.}

Вывод является буквами, ascii-код которых дан на вводе (32-й символ таблицы является пробел).

Асимптотика: $O(1)$.



\subsubsection*{D. Digits}

\problemauthor{ Абдикалыков А.К.}

Пусть $d[n][k]$ --- количество $n$-значных чисел, у которых сумма цифр равна $k$. Ясно, что если у числа отбросить последнюю цифру $z$, то получим число с суммой цифр $k - z$. Значит,
$$d[i][j] = \sum_{z=0}^{\min(9,j)} d[i-1][j-z].$$
Что легко просчитать от для всех $i$ от 1 до $n$ и $j$ от 1 до $k$. Начальные значения $d[0][0] = 1$ и $d[i][0] = 0$.

Асимптотика: $O(n k)$.



\subsubsection*{E. Elimination}

\problemauthor{ Баев А.Ж.}

Пусть $d[i][j]$ --- количество работающих участков у $i$-й жилы среди участков с $(j-m+1)$-го до $j$-го включительно, которые можно вычислить за $O(1)$ каждый:$$d[i][j] = d[i][j-1] + a[i][j] - a[i][j-m]$$
для всех $i$ от 1 до $k$ и $j$ от $m$ до $n$.
Ответом на задачу будет $\max_{m\leqslant j \leqslant n} c_j$,
где $c_j$ --- количество $d[i][j] = m$, для всех $i$ от 1 до $k$.

Асимптотика: $O(n k)$.



\subsubsection*{F. Forts}

\problemauthor{ Баев А.Ж.}

\begin{center}
\definecolor{light}{rgb}{0.85,0.85,0.85}
\begin{tikzpicture}[x=6, y=6]
\begin{scope}
  \clip (-10,-10) -- (-10,0) -- (0,0) -- (0, -10);
  \fill[fill=light, draw=none] (3, 4) circle (10);
\end{scope}
\draw (-7, 4) arc (180:270:10);
\draw (-10, 0) -- (0, 0);
\draw (0, -10) -- (0, 0);
\draw (3, 4) -- (0, 0);
\draw (3, 4) -- (-6.165, 0);
\draw (3, 4) -- (0, -5.539);
\node at (3, 4.5) {O};
\node at (-7, -1) {A};
\node at (-1, -7) {B};
\node at (-1, -1) {C};
\end{tikzpicture}
\end{center}

Искомая площадь равна нулю, если $a^2 + b^2 > r^2$. Иначе ее можно найти как разность площали кругового сектора $OAB$ и двух треугольников $AOC$ и $BOC$.

Асимптотика: $O(1)$.

\head{Весенний тур 2017}{31 мая 2017}
\input{solutions/2017-05}

\head{Осенний тур 2017}{28 октября 2017}
\input{solutions/2017-10}

\head{Зимний тур 2017}{19 декабря 2017}
\input{solutions/2017-12}

\head{Весенний тур 2018}{19 мая 2018}
\input{solutions/2018-05}

\newpage

\section{Тематический указатель}

Линейный поиск и реализация
\begin{itemize}
\item A. Ayat and the film (2014 весна)
\item D. Domino (2015 осень)
\item F. Footprints (2015 осень)
\item G. Game of Castles (2015 зима)
\item A. Alexandra’s subtractions (2016 весна)
\item E. Examination aura (2017 весна)
\item I. Incalculable result (2017 весна)
\item C. Course (2017 зима)
\end{itemize}

Автомат
\begin{itemize}
\item B. Bekarys and khet (2018 весна)
\end{itemize}

Конструктив
\begin{itemize}
\item I. Interesting permutation (2014 зима)
\item A. Automultiplicative numbers (2014 зима)
\item A. Аdamant digit (2015 весна)
\item A. Alternative result (2015 осень)
\item A. Alexandra and Exam (2015 зима)
\item C. Calculation of Erulan Numbers (2015 зима)
\item A. A sad number (2016 осень)
\item B. Brute force (2016 осень)
\item H. Hobby (2016 осень)
\item H. Hyper numbers (2018 весна)
\end{itemize}

Математика
\begin{itemize}
\item G. GOR (2014 зима)
\item C. Car collection (2015 осень)
\item D. Do Rain Dance (2015 зима)
\item K. Keg and dipper (2016 осень)
\item L. Lazy programming (2016 осень)
\item B. Bead garland (2016 осень)
\item A. A train problem (2016 зима)
\item G. Good round numbers (2017 весна)
\item A. Appetizing problem (2017 осень)
\item B. Bekarys’ problem (2017 осень)
\item A. Askhana (2017 зима)
\item A. Azat and bookshelf (2018 весна)
\end{itemize}

<<Разделяй и властвуй>>, бинарный поиск
\begin{itemize}
\item G. Great graph (2015 весна)
\item I. Insidious time limit (2015 весна)
\item J. Jagged sequence (2015 весна)
\item E. Enlarged triangle (2015 осень)
\item E. Experiment with tea (2016 весна)
\item I. IMC problem (2016 осень)
\item E. Examination aura (2017 весна)
\item F. Fibonaccissimo (2017 весна)
\item E. Easy problem (2017 осень)
\end{itemize}

Два указателя:
\begin{itemize}
\item F. Fix position (2017 зима)
\end{itemize}

Сортировка и подсчет 
\begin{itemize}
\item J. Jagged sequence (2015 весна)
\item E. ECM and Cognitive Dissonance (2015 весна)
\item D. Decks (2014 зима)
\item C. Curtains (2016 осень)
\item E. Easy shifting (2017 зима)
\end{itemize}

Предпросчет
\begin{itemize}
\item G. Golden problem (2017 осень)
\end{itemize}

Алгоритмы на графах
\begin{itemize}
\item B. Beautiful tree (2013 зима)
\item H. Ha-ha-ha (2014 весна)
\item H. Holes (2014 зима)
\item G. Great graph (2015 весна)
\item E. Experiment with tea (2016 весна)
\item E. Excursion in snowy cube (2016 осень)
\item J. Join the knowledge (2016 осень)
\item D. Diners (2017 весна)
\item D. Dice problem (2017 осень)
\item F. Flat problem (2017 осень)
\item G. Graphland (2017 зима)
\end{itemize}

Геометрия
\begin{itemize}
\item D. Difficult geometry (2013 зима)
\item H. Hypnoses (2013 зима)
\item G. Geometry (2014 весна)
\item E. Ellipse (2014 зима)
\item E. Empty сornet (2015 весна)
\item E. Enlarged triangle (2015 осень)
\item B. Ball and Snowy Cube (2015 зима)
\item J. Jelly cake (2015 весна)
\item F. Forts (2016 зима)
\item C. Circles (2017 весна)
\item H. Honey cake problem (2017 осень)
\item G. Giant chandelier (2018 весна)
\end{itemize}

Разложение на простые
\begin{itemize}
\item E. Easy number (2013 зима)
\item D. Dima’s divided numbers (2014 весна)
\item F. Fibonacci and Prime Number (2015 зима)
\item D. DPK rover (2018 весна)
\end{itemize}

Решето Эратосфена
\begin{itemize}
\item G. Great graph (2015 весна)
\item G. Great divisors (2015 осень)
\item H. Honest gifts (2015 осень)
\item D. Doubtful numbers (2016 весна)
\end{itemize}

Алгоритм Евклида
\begin{itemize}
\item F. Friends (2013 зима)
\item D. Deep tree (2015 весна)
\item B. Billiards (2017 весна)
\end{itemize}

Комбинаторика и теория вероятностей
\begin{itemize}
\item C. Cube (2013 зима)
\item B. BNF (2014 зима)
\item C. Cheer up! (2014 зима)
\item G. Glowing letters (2016 весна)
\end{itemize}

Системы счисления
\begin{itemize}
\item G. GOR (2014 зима)
\item F. Fantastic system (2015 весна)
\item B. Book of all the words (2016 весна)
\item D. Dimitriy and broken sum (2016 осень)
\item A. Azats rounding (2017 весна)
\end{itemize}

Длинная арифметика
\begin{itemize}
\item E. Elegant system (2014 весна)
\item B. Binary palindromes (2015 весна)
\end{itemize}

Строки
\begin{itemize}
\item A. A (2013 зима)
\item C. Comparing (2014 весна)
\item C. Composition of matrices (2015 весна)
С. Change the word (2016 весна)
\item F. Forced reduction (2018 весна)
\end{itemize}

Динамическое программирование
\begin{itemize}
\item G. Game (2013 зима)
\item F. Fantastic chess (2014 весна)
\item C. Cheer up! (2014 зима)
\item F. Fantastic system (2015 весна)
\item I. Inner subset (2015 осень)
\item H. Harmonic permutations (2016 весна)
\item I. Infinity problem (2016 весна)
\item F. Food getting ways (2016 осень)
\item D. Digits (2016 зима)
\item E. Elimination (2016 зима)
\item C. Car showroom problem (2017 осень)
\item B. Binecraft (2017 зима)
\item D. Decoration (2017 зима)
\item C. Computer vision (2018 весна)
\end{itemize}

Задача без условия
\begin{itemize}
\item B. Big dipper (2014 весна)
\item F. Flip (2014 зима)
\item H. H (2015 весна)
\item B. Boolean (2015 осень)
\item F. Five words (2016 весна)
\item G. Great and mighty (2016 осень)
\item C. Champion (2016 зима)
\item H. Hit a ball (2017 весна)
\item I. Is that even a problem? (2017 осень)
\item E. Excellent idea (2018 весна)
\end{itemize}
\end{document} 
