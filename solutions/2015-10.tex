\subsubsection*{A. Alternative result} 

\problemauthor{Абдикалыков А.К.}

Несложно убедиться, что можно получить все значения от 0 до $3n$, кроме $3n-1$.

Асимптотика: $O(n)$.



\subsubsection*{B. Boolean} 

\problemauthor{Абдикалыков А.К.}

Необходимо было вывести $N$-е слово из текста.

Асимптотика: $O(1)$.



\subsubsection*{C. Car collection} 

\problemauthor{Баев А.Ж.}

Ответом на задачу является:
$$\sum_{i=1}^{n-1} \sum_{j=i+1}^{n} a_i a_j = \frac{1}{2} \left( (\sum_{i=1}^{n} {a_i})^2 - \sum_{i=1}^{n} a_i^2 \right).$$

Асимптотика: $O(n)$.

Замечание: наивное решение не проходит ограничения по времени.



\subsubsection*{D. Domino} 

\problemauthor{Баев А.Ж.}

Промоделируем падения слева направо и справа налево. Для этого найдем максимальную длину положительной подстроки массива $l_i$, где $l_i = a_i - a_{i-1} - h_{i-1}$, и максимальную длину положительной подстроки массива $r_i$, где $r_i = a_i - a_{i+1} - h_{i+1}$. Ответов будет максимум из первого и второго случая.

Асимптотика: $O(n)$.



\subsubsection*{E. Enlarged triangle} 

\problemauthor{Баев А.Ж.}

Пусть $S(a, b, c)$ --- площадь треугольника со сторонами $a$, $b$, $c$. Несложно проверить, что функция $f(m) = S(a + m, b + m, c + m)$ является монотонно возрастающей (при условии, что $m > 0$ и треугольник с данными сторонами существует). Значит, ответ можно найти бинарным поиском по $m$ на отрезке $[0; \sqrt{2 S}]$.

Асимптотика: $O(\log S)$.



\subsubsection*{F. Footprints} 

\problemauthor{Баев А.Ж.}

Обозначим начальную позицию (0, 0). Далее промоделируем шаги $(x_i, y_i)$. Минимальные размеры лабиринта будут $(\max\limits_{1 \leqslant i \leqslant n} x_i - \min\limits_{1 \leqslant i \leqslant n} x_i)$ и $(\max\limits_{1 \leqslant i \leqslant n} y_i - \min\limits_{1 \leqslant i \leqslant n} y_i)$ соответственно.

Асимптотика: $O(N)$.



\subsubsection*{G. Great divisors} 

\problemauthor{Абдикалыков А.К.}

Максимальный собственный делитель числа $n$ равен $n / p_n$, где $p_n$ --- минимальное простое число, на которое делится $n$. Последовательность $p_n$ легко построить, используя стандартный алгоритм решета Эратосфена (у всех еще не вычеркнутых чисел вида $p^2 + p \cdot j$ минимальным простым делителем будет $p$).

Асимптотика: $O(n \log n)$.



\subsubsection*{H. Honest gifts} 

\problemauthor{Баев А.Ж.}

Ясно, что максимальным количество наборов с общим количество $p$ синих и $q$ красных карандашей будет $(p, q)$ --- наибольший общий делитель $p$ и $q$. Поэтому достаточно перебрать все числа $i$ от 0 до $k$ и выбрать максимум из $gcd(a - i, b - (k - i))$.

Асимптотика: $O(k \log \max(a, b))$.



\subsubsection*{I. Inner subset} 

\problemauthor{Баев А.Ж.}

Необходимо посчитать количество способов выбрать подпоследовательность так, чтобы сумма чисел была кратна $k$. Обозначим $d[i][r]$ --- количество подпоследовательностей из первых $i$ элементов, которые в сумму дают остаток $r$ при делении на $k$. Каждое такое подножество можно получить, либо добавив $a[i]$ элемент к подножествами множества из первых $i-1$ с остатком суммы равным $(r - a[i]) \bmod k$, либо не добавляя $a[i]$ элемент:
$$d[i][r] = d[i-1][r] + d[i-1][(r - a[i]) \bmod k].$$
Инициализировать динамику можно $d[0][0] = 1$ и $d[0][r] = 0$ при $r$ от 1 до $k-1$.

Асимптотика: $O(n k)$.

Замечание: не стоит забывать производить каждое сложение по модулю $10^9 + 7$, иначе произойдет переполнение ответа.