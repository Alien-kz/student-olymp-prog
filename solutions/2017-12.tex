\subsubsection*{A. Askhana}

\problemauthor{Бекмаганбетов Бекарыс (ММ-2016).}

Обозначим $$A = \left[ \frac{N}{3} \right] \Bigl(1000 (Q_1 + Q_2 + Q_3) - (P_1 + P_2 + P_3) \Bigr).$$

Тогда ответ равен:
$$
\begin{cases}
A, & N \equiv 0 \bmod 3,\\
A + 1000 Q_1 - P_1, & N \equiv 1 \bmod 3,\\
A + 1000 (Q_1 + Q_2) - (P_1 + P_2), & N \equiv 2 \bmod 3.\\
\end{cases}
$$

Асимптотика: $O(1)$. 

Решение за $O(N)$ было лишь частичным.

\subsubsection*{B. Binecraft}

\problemauthor{Шарипов Азат (ВМК-2016).}


Обозначим $d_{i,j}$ --- количество свободных клеток в прямоугольнике $(0; 0) - (i; j)$. Вычислить все $d_{i,j}$ можно за $O(W \cdot H)$ с помощью формулы включения-исключения:
$$
d_{i,j} = 
\begin{cases}
d_{i-1, j} + d_{i, j-1} - d_{i, j}, & \text{ если } a_{ij} \text{ --- занята,}\\
d_{i-1, j} + d_{i, j-1} - d_{i, j} + 1, & \text{ если } a_{ij} \text{ --- свободна.}\\
\end{cases}
$$

Для каждой клетки $(i, j)$ исходной парковки можно за $O(1)$ проверить, сколько свободных клеток находится в прямоугольнике $N \times M$ c правым нижним углов в $(i, j)$:
$$d_{i,j} - d_{i - N, j} - d_{i, j - M} + d_{i - N, j - M}.$$
Если эта величина равна $N \cdot M$, то данное расположение подходит (прибавляем к ответу 1). В случае, если машина не квадратная ($N \neq M$), то выполняем такие же действия для прямоугольника $M \times N$.

Асимптотика: $O(W \cdot H)$. 

Решение за $O(N \cdot M \cdot W \cdot H)$ было лишь частичным.

\subsubsection*{C. Course}


\problemauthor{Шарипов Азат (ВМК-2016).}

Сделаем предпросчет для каждого дня {\tt d}: {\tt t[i]} --- время прихода {\tt i}-го студента в день {\tt d}. Отсортируем массив пар {\tt (t[i], i)} (не более 36 пар) и заполним массив ответов {\tt ans[d][r][c] = t[6 * (r - 1) - (c - 1)]}. Далее отвечаем на каждый запрос за $O(1)$.

Асимптотика: $O(M + D)$. 

Решение за $O(M \cdot D)$ было лишь частичным.

\subsubsection*{D. Decoration}


\problemauthor{Седякин Илья (ВМК-2013).}

Посчитаем аккумулированные префиксные суммы $s_m = \sum_{i=0}^m a_i$, при $m$ от 0 до $N$. Заметим, что сумма $a_{i} + ... + a_{j} = s_j - s_{i-1}$ делится на $K$ тогда, и только тогда, когда $s_j$ и $s_{i-1}$ дают одинаковые остатки. 

Посчитаем, $d_r$ --- количество префиксных сумм, которые дают в точности остаток $r$ при делении на $K$ (с подсчетом перебрав все префиксные суммы) при $r$ от 0 до $K-1$. 

Ответ легко найти за $O(K)$: $$\sum\limits_{r = 0}^{K-1} \frac{d_r(d_r - 1)}{2}.$$

Асимптотика: $O(N + K)$. 

Решение за $O(N^2)$ было лишь частичным.

\subsubsection*{E. Easy shifting}


\problemofferer{Шарипов Азат (ВМК-2016).}

Число инверсий данной перестановки можно вычислить за $O(N \log N)$ с помощью известной модификации сортировки слиянием. Посчитаем как изменится число инверсий после циклического сдвига влево.

Допустим дана некоторая перестановка, в которой известно $P$ --- число инверсий. Обозначим $X$ --- первое слева число перестановки. Если убрать число $X$ из перестановки, то количество инверсий уменьшится на $(X - 1)$ (все числа меньшие $X$). Если добавить число $X$ справа, то добавится $(N - X)$ инверсий. Таким образом общее количество инверсий станет $$P - 2 X + 1 - N.$$ 
То есть за $O(1)$ можно узнать на сколько изменится число инверсий при циклическом сдвиге влево. Максимальное число инверсий среди всех сдвигов можно найти перебором всех элементов массива без явного сдвига (за $O(N)$).

Асимптотика: $O(N \log N)$. 

Решение за $O(N^2)$ было лишь частичным.

\subsubsection*{F. Fix position}


\problemofferer{Аскергали Ануар (ВМК-2016).}

Заметим, что если расстояние между $i$-й точкой первого ряда и $j$-й точкой второго ряда минимальное, то и разница координат по оси $OX$ между этими точками будет минимальная (так как разница координат по оси $OY$ равна 1). 

Отсортируем точки в первом и втором ряду за $O(N \log N + M \log M)$. Обозначим $u_i$ и $d_j$ полученные координаты в первом и втором ряду соответственно.

\textbf{Решение 1.} Для каждой точки $u_i$ из первого ряда запустим целочисленный тернарный поиск минимума для выпуклой вниз функции $f(j) = |u_i  - d_j|$.

Асиптотика: $(N + M) \log (M N)$.

\textbf{Решение 2.} Заметим, что если $u_i < u_{i+1} < d_j$, то $|u_i - d_j|> |u_{i+1} - d_j|$. Заведем 2 указателя $i = 0$ и $j = 0$. Если $u_{i+1} < d_{j + 1}$ то двигаем указатель $i = i + 1$, иначе двигаем указатель $j = j + 1$. Среди всех таких пар находим минимальную. Таким образом за $O(N + M)$ найдем ответ с помощью двух указателей.

Асиптотика: $O(N \log N + M \log M)$.

Решение за $O(N \cdot M)$ было лишь частичным.

\subsubsection*{G. Graphland}


\problemauthor{Бекмаганбетов Бекарыс (ММ-2016).}


Построим компоненты связности обходом в ширину или в глубину за $O(N + M)$. Для каждой вершины запомним номер компоненты связности, в которой она находится. Далее переберем все ребра, которые можно добавить и которые при этом соединяют вершины из разных компонент связности. Среди таких ребер выберем наиболее дешевое.

Асиптотика: $O(N + M + T)$.